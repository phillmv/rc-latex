%%%%%%%%%%%%%%%%%%%%%%%%%%%%%%%%%%%%%%%%%
% Thin Sectioned Essay
% LaTeX Template
% Version 1.0 (3/8/13)
%
% This template has been downloaded from:
% http://www.LaTeXTemplates.com
%
% Original Author:
% Nicolas Diaz (nsdiaz@uc.cl) with extensive modifications by:
% Vel (vel@latextemplates.com)
%
% License:
% CC BY-NC-SA 3.0 (http://creativecommons.org/licenses/by-nc-sa/3.0/)
%
%%%%%%%%%%%%%%%%%%%%%%%%%%%%%%%%%%%%%%%%%

%----------------------------------------------------------------------------------------
%	PACKAGES AND OTHER DOCUMENT CONFIGURATIONS
%----------------------------------------------------------------------------------------

\documentclass[letterpaper,12pt]{article} % Font size (can be 10pt, 11pt or 12pt) and paper size (remove a4paper for US letter paper)


\usepackage[protrusion=true,expansion=true]{microtype} % Better typography
\usepackage{graphicx} % Required for including pictures
\usepackage{wrapfig} % Allows in-line images
\usepackage[labelfont=bf]{caption}

% footnotes at the bottom
\usepackage[bottom]{footmisc}

% quotes
\usepackage [autostyle=false]{csquotes}
\MakeOuterQuote{"}


% LuaTeX fonts:
\usepackage{fontspec}
\setmainfont{Times New Roman}
\setsansfont{TeX Gyre Heros} % Helvetica
% pdftex shit
% \usepackage{mathpazo} % Use the Palatino font
% \usepackage{times} % use times new roman
% \setmainfont{Times New Roman}
% \usepackage{helvet}

% for changing section & subsection fonts
\usepackage{titlesec}

% i want caption size to be smaller
\captionsetup{font=footnotesize}

% get section and subsection to be in helvet
\titleformat{\section}
  {\normalfont\sffamily\Large\bfseries}
  {\thesection}{1em}{}

\titleformat{\subsection}
  {\normalfont\sffamily\normalsize\bfseries}
  {\thesection}{}{}

% list references in table of contents
\usepackage[nottoc,numbib]{tocbibind}

% control line spacing
% \usepackage{setspace}

% \usepackage[T1]{fontenc} % Required for accented characters
% \usepackage[utf8]{inputenc}
% \linespread{1.05} % Change line spacing here, Palatino benefits from a slight increase by default

% spreadsheets
% \usepackage{array,booktabs}
% \usepackage{csvsimple}
% \usepackage{datatool}
\usepackage{array,booktabs}
% \usepackage{lscape}
% \usepackage{rotating}


% for handling margins on pages with tables
\usepackage[letterpaper]{geometry}
\geometry{textwidth=385pt, textheight=620pt}
% \usepackage{showframe}
\usepackage{layouts}

% configure links
% IMPORT IT LAST
\usepackage{hyperref}

\hypersetup{
    colorlinks=true,
    allcolors=blue
}

% list of tables AND figures
\renewcommand\listfigurename{List of Figures and Tables}


\makeatletter
\renewcommand\@biblabel[1]{\textbf{#1.}} % Change the square brackets for each bibliography item from '[1]' to '1.'
\renewcommand{\@listI}{\itemsep=0pt} % Reduce the space between items in the itemize and enumerate environments and the bibliography

\renewcommand{\maketitle}{ % Customize the title - do not edit title and author name here, see the TITLE block below
\begin{flushright} % Right align
{\LARGE\@title} % Increase the font size of the title

\vspace{50pt} % Some vertical space between the title and author name

{\large\@author} % Author name
\\\@date % Date

\vspace{40pt} % Some vertical space between the author block and abstract
\end{flushright}
}

%----------------------------------------------------------------------------------------
%	TITLE
%----------------------------------------------------------------------------------------

\title{\textsf{\textbf{Actually, Rent Control Is Great}\\ % Title
Revisiting Ontario's Experience, the Supply of Housing, and Security of Tenure}} % Subtitle

% \author{\textsc{Phillip Mendonça-Vieira} % Author
% \\{\textit{Federation of Metro Tenants Association}}} % Institution
%

\author{\textsc{Phillip Mendonça-Vieira}\footnote{ This research was conducted independently during the fall of 2017 and was funded by the author's personal savings. The author can be reached at phillmv@okayfail.com}}
\date{\today} % Date

%----------------------------------------------------------------------------------------

\begin{document}

\maketitle % Print the title section

%----------------------------------------------------------------------------------------
%	ABSTRACT AND KEYWORDS
%----------------------------------------------------------------------------------------

% ensure abstractname is also in helvet
\renewcommand{\abstractname}{\sffamily{Abstract}} % Uncomment to change the name of the abstract to something else

% \begin{abstract}
% Morbi tempor congue porta. Proin semper, leo vitae faucibus dictum, metus mauris lacinia lorem, ac congue leo felis eu turpis. Sed nec nunc pellentesque, gravida eros at, porttitor ipsum. Praesent consequat urna a lacus lobortis ultrices eget ac metus. In tempus hendrerit rhoncus. Mauris dignissim turpis id sollicitudin lacinia. Praesent libero tellus, fringilla nec ullamcorper at, ultrices id nulla. Phasellus placerat a tellus a malesuada.
% \end{abstract}

\begin{abstract}
  Rent controls are criticized for acting as a severe disincentive to new and existing rental construction. From 1992 to 2017 the province of Ontario exempted all new buildings from its rent control regime. What was the effect on rental construction in Ontario during that time period? Finding that rental construction continues to be depressed, this paper documents contemporary Canadian housing policy initiatives and investigates the theoretical and empirical record of rent controls in other jurisdictions. This paper then argues that, rather than targetting affordability, rent controls regulate the provision of security of tenure -- which should be seen as a right of tenants as well as homeowners.
\end{abstract}

\hspace*{3,6mm}\textit{Keywords:} rent control, security of tenure, housing, Ontario, Canada % Keywords

\vspace{30pt} % Some vertical space between the abstract and first section

%----------------------------------------------------------------------------------------
%	ESSAY BODY
%----- q-----------------------------------------------------------------------------------

% \singlespacing
{\sffamily\tableofcontents}
\setcounter{secnumdepth}{-2}

\let\oldnumberline\numberline% Copy \numberline into \oldnumberline
\renewcommand{\numberline}[1]{\hspace*{-1.5em}}% Remove number argument
\listoffigures
\let\numberline\oldnumberline% Restore \numberline (if needed)

% \onehalfspacing % spacing!!!
% \setlength{\parskip}{0.5em} % space between paras

% \clearpage
% {\csvautotabular{ontario-housing-starts-1968-2016.csv}}
% \begin{landscape}
% \begin{sidewaystable}
% \DTLloaddb{stores}{ontario-housing-starts-1968-2016.csv}
% \DTLdisplaydb{stores}
% \end{landscape}
% \end{sidewaystable}

% \begin{tabular}{l*{6}{c}r}
% Team              & P & W & D & L & F  & A & Pts \\
% \midrule
% Manche,ster United & 6 & 4 & 0 & 2 & 10 & 5 & 12  \\
% Celtic            & 6 & 3 & 0 & 3 &  8 & 9 &  9  \\
% Benfica           & 6 & 2 & 1 & 3 &  7 & 8 &  7  \\
% FC Copenhagen     & 6 & 2 & 1 & 3 &  5 & 8 &  7  \\
% \end{tabular}
\clearpage
\currentpage
\pagedesign
% \pagevalues
\section{Introduction}

In April, the government of Ontario decided to extend rent control to every dwelling in the province, as opposed to just buildings constructed before 1991.

Different jurisdictions have overlapping definitions of "rent control". What Ontario engages in is described by some academics and regions as a tenancy rent control, or rent stabilization. It works like this: once a year, your landlord can raise your monthly rent only up to a guideline pegged to inflation, or inflation plus 3\% if their costs have spiked or they made improvements to the unit. Between tenants, landlords are free to price rents at whatever the market will bear.\cite{ltb-2017}

This prompted a lot of commentary, ranging from \href{http://tvo.org/article/current-affairs/the-next-ontario/ontario-needs-a-rental-rethink-but-should-tread-carefully}{benign skepticism} to \href{https://beta.theglobeandmail.com/real-estate/toronto/new-ontario-rent-control-rules-exact-opposite-of-what-is-needed-analyst-warns/article34569276/}{vigorous condemnation}. The chorus sounded like this:

Against all common sense, the province is handing its cities a poisoned chalice: it is textbook economics that price controls sharply reduce the value of new construction.\cite{gee-2017}  Under rent control, the quantity and quality of available rental units will fall as developers are less incentivized to build or invest in rental properties --- all of which exacerbates any price crunch.\cite{tal-2017} The fact is, rent control would largely help high-end renters in a high-end market, since most units built after the rent control exemption are condos. It’s tough to see how rent control would accomplish much except transferring money from unit owners to their tenants.\cite{selley} Instead, the province should be tackling the root of the problem: the supply of new housing units in Toronto and elsewhere is not keeping up with demand.\cite{thesun}



\section{The case against rent control}
\subsection{Vanishing rental supply}


\subsection{Controls in Ontario}

\begin{figure}[ht]
\centering 
\includegraphics[width=\textwidth]{"Ontario Housing Starts 1969-1986".png}
\caption[Ontario housing starts by intended market 1969-1986]{Ontario housing starts by intended market 1969-1986. This graph does not distinguish between private and government assisted rentals: from 1969-1974 private rentals constituted 72\% of all starts, but from 1975 onwards they were under half of all rentals.}
\label{fig:gallery} 
\end{figure}

\section{Revisiting Ontario's experience}



\begin{figure}[ht]
\centering 
\includegraphics[width=\textwidth]{"Ontario Housing Starts 1969-2016".png}
\caption[Ontario housing starts by intended market 1969-2016]{Ontario housing starts by intended market 1969-2016. Data from 1987 onwards is restricted to areas with over 10,000 people, and therefore undercounts total starts.}
\label{fig:gallery} 
\end{figure}



\section{A theoretical overview}
\subsection{The many types of controls}

The point of this paper isn't to say that price controls are a paragon of virtue, but instead to critically examine the received wisdom: the picture is rather murkier than that shown in in the standard textbook analysis.

One major problem with this discussion is that there are actually many qualitatively different kinds of rent controls, which different regions have experimented with over different periods of time. When designing a control, we can decide when and how prices increase, whether they track inflation or another indicator, whether capital expenditures can be recovered, at what rate vacant units can reset to what the market will bear (if at all), and so on.\footnote{\label{^nota-bene-2} For a more exhaustive treatment of the literature, see Lind (2001) and (2003), Brescia (2005), Jenkins (2009), Grant (2011) and Ambrosius et al (2015)}

Richard Arnott, in a widely read paper from 1995, broadly categorizes rent control policy into two distinct phases.\footnote{\label{^arnott-original} This distinction between "first" and "second" generation or "hard" and "soft" controls is not original to Arnott; Hulchanski (1984) also refers to them that way and it's likely that categorization was made contemporaneously when these controls were enacted in the 1970s.} The first wave, or generation, of controls were enacted around World War II and they imposed nominal rent freezes tied to individual units. In a planned war economy that lacked much in the way of private housing construction, this made a certain amount of sense; but afterwards most regions dismantled their controls and only New York City (and some European cities) maintained that wartime policy.\cite{^arnott-1995}

The second generation began in the 1970s as rent control ordinances were passed in Boston, Los Angeles, San Francisco and in a variety of towns in California, Massachussetts, New Jersey, New York, etc. The structure of Canadian governance saw these policies accumulate at the provincial level, and during this period ten provinces also enacted rent controls (though most have since abandoned them). These policies "differed significantly from the first-generation rent control programs" and usually allowed automatic increases in rent, the passing through of additional costs, and other like-minded provisions.\cite{^arnott-1995}

Separating controls into "hard" and "soft" is not especially useful, though. Hans Lind, writing in 2001, goes on to define five distinct functional types of rent control. In his view, regulations can protect sitting tenants from being charged above-market rents (type A), or from increases in rent unattached to increases in costs (type B). Alternatively, regulations can instead bind to units and prevent landlords anywhere from charging above-market rents (type C), prevent rapid inflation by smoothing increases (type D) or, finally, prevent rents from ever reaching actual market prices (type E).\footnote{\label{^note-on-lind}  I am not sure that Lind's categories are especially useful for comparing different regulatory systems, in so far that its abstraction elides too many relevant details. However, it's great for illustrating the wide variety in intent and implementation.} \cite{^lind-2001}

The takeaway here is that when we talk about regulations, we have to be specific since their goals, mechanisms and therefore impacts are going to be different. New York City's rent controls, which Lind identifies as an extreme version of a type E control,\cite{^lind-2001} are the most famous and well studied example, and consequently critics are quick to conflate all rent regulation with the kind experienced there.\footnote{\label{^widely-cited-2} For example, both the \emph{Globe}'s Marcus Gee\cite{^gee-2017} and CIBC's Benjamin Tal\cite{^tal-2017} cite New York City when writing their op-eds.} That tendency is unfortunate, since in doing so they perform a sleight of hand: New York City's complex and overlapping rent regulations were enacted at different points of time,\footnote{\label{^nycrgb-2016-note} NYC distinguishes between rent \emph{controls}, which target buildings built before 1947 and continuous tenancy prior to 1971, and rent \emph{stabilization}, which targets buildings built prior to 1974 with rents under \$2,700. Different tenants under different systems have different rights. Rent \emph{controls} may have been a nominal freeze when they were enacted, but today landlords are entitled to a 7.5\% increase per annum.\cite{^nycrgb-2016}\cite{^curbed-2017}} and therefore its experience is idiosyncratic and unlikely to be directly applicable to other cities.

Of course, they're not wrong to criticize nominal rent freezes: obviously, a control regime that over time lowers real income below that of real expenditures is a \emph{bad idea}. It transforms rental properties into endless money pits. It's fine, and likely necessary, to subsidize some or other aspect of how we produce or provide housing units in order to achieve our policy goals -- but it's unreasonable to expect that subsidy to be provided to the exclusive detriment of individual landlords. If investors wish to transfer their wealth to tenants they don't need to go through the trouble of erecting a building.  

But that's a false choice. We're not limited to choosing between an unfettered market and a ruinously restrictive price control.

\subsection{In a competitive market, prices don't increase arbitrarily}

When economists discuss losses in efficiency, allocation and welfare, they're comparing our real world with an idealized "perfectly competitive" market where landlords compete to produce homogeneous housing units, there are no externalities, every actor possesses perfect information, and so on. In this view, a landlord faced with increased demand is free to increase prices and profits accordingly. Abnormally high profits, though, attract other potential landlords who by virtue of adding to the supply of apartments will then drive down their prices.

Given a perfect market any sketch on a napkin will show that if prices are not allowed to rise with demand then new entrants will stay put, supply will not increase, and shortages will follow. However, given perfect competition, the market prices any landlord can fetch will over the long run equal their marginal cost, i.e. the amortized cost of building and operating a housing unit plus the landlord's opportunity cost. Put another way, in a \emph{well functioning market} it's more or less unreasonable to expect the rents any given landlord is able to extract will grow much faster than costs and inflation.\footnote{\label{^arnott-monopoly}  Arnott makes the case that housing markets are actually monopolistically competitive since housing structures and preferences are not homogenous, there are substantial asymmetries in information, and transactions costs are non-trivial. Consequently, rents are set higher than their efficient level and the corresponding deadweight loss can be mitigated (Arnott 2003, p. 106). But for our purposes we don't need to engage with this argument.}

The implication here is that while absolute price ceilings can be and are harmful there is no obvious reason why price \emph{smoothing} such that increases match but do not exceed costs should have a strong effect on the incentive to create new rental housing.\footnote{\label{^arnott-2003-note} "The introduction of tenancy rent control has no obviously strong effect on the incentives to undertake rental housing construction."\cite{^arnott-2003}} Other economists have reached this conclusion. Writing in 2001, Alastair McFarlane developed an econometric model of rent stabilization and concluded that "because allowing fully flexible base rents permits landlords to capture all of the advantages of a rent growth control, neither the timing nor the density of development will be affected by rent stabilization", though landlords are incentivized to redevelop sooner than later.\cite{^macfarlane-2001}

Trivially, investment in multi-residential buildings is a function of one's cost of equity, cost of financing and net operating income; developers may invest in a project expecting significant growth in their net operating income, but in a competitive market that is a rather risky assumption. Therefore, even in the absence of rent controls, projects by and large must be cashflow positive given rents available immediately post-construction.\cite{^black-2012-4}

As far as new supply is concerned we can therefore conceive of a non-harmful rent control: if prices increase with costs and inflation, landlord cashflows should largely be unaffected. It's easy to see why: a cost-adjusted tenancy rent control primarily impacts only one area of the development process: the initial lease up of an empty building.\cite{^tait-2017} Given the long-term nature of multi-residential investment, and provided with the ability to adjust for initial mistakes, over the long run the impact on their finances should be reasonable if not minimal -- and their business model can be satisfied so long as cashflows keep up with costs and capital expenditures.

You don't have to take my word for it. Take GWL Realty Advisors, whose president Paul Finkbeiner was quoted in the \emph{Financial Post}:

\begin{quote}
  “We believe there is a strong demand for rental apartments and this property will lease up over time,” Finkbeiner said about his Livmore project, [...] “Apartments provide good long-term returns and very low vacancy levels, it’s just one of the best assets classes from a stability point of view.”
\[...\]
GWL seems to think it can work within the new provincial rules. “As a developer, we are building something that will last for 25-50 years that works for tenants,” said Finkbeiner. “We want long-term renters which is also consistent with our investors that are long-term in nature. These buildings go to pay pensions and people’s investments.”

 He noted Ontario still allows rents to be raised to market level once a tenant leaves a unit and capital improvements to buildings can also be passed on to tenants.

“All we want is a fair rent for our apartments, we do not want above guidelines. We have been able to work within rent controls and still deliver a good product for our investors and tenants,”\cite{^marr-2017}
\end{quote}

\subsection{Evidence on supply from other jurisdictions}

Recall that Ontario's current rent control regime pegs rents to inflation, does not control rents between tenants, and allows cost pass through. Lind categorized it as a type B control,\cite{^lind-2001} and elsewhere it is variously called a kind of tenancy rent control or rent stabilization.

We saw earlier that though Ontario's previous rent control regimes may have been harmful, they were unlikely to be the main disincentives acting on supply -- and especially so since 1998, when vacancies were decontrolled and new construction was exempted entirely. Since we need to compare apples-to-apples, what other evidence can we draw for the impacts of type B rent controls?

Consider Manitoba, whose regulation scheme is broadly similar\footnote{\label{^note-on-mb}  Manitoba does not allow the rent in buildings with more than 3 units to be reset on vacancy, but it exempts new buildings for 20 years and has a more generous cost pass through provision. Grant (2011)} and has regulated rents since roughly 1976. Hugh Grant, writing in 2011, argues that there is no evidence that Manitoba's rent regulation program had a negative effect on the supply of new, or maintenance of existing, rental properties. Manitoba at the time was experiencing a low vacancy rate, which Grant attributed to a rapid influx of immigration, and a relatively inelastic supply due to large planning-to-completion time lags and uncertainty about future rates of population growth.\cite{^grant-2011}

In New Jersey, over one hundred municipalities have enacted their own rent controls. Each city implemented their regulation differently, but by and large they all permit automatic increases, passing on capital improvements, etc; almost half also engage in vacancy decontrol. In 2015 Joshua Ambrosius et al used the 2010 United States Census and compared the regulated cities with unregulated cities. They found that, once they controlled for other factors, New Jersey rent control policies had no statistical impact on rental quality, rental supply, property appreciation or foreclosure rates in the cities that enacted them.\cite{^ambrosius-2015}

In fact, tenancy rent controls seem to barely control rents at all. Earlier this year Graham Haines analyzed Ontario's rent regulations and developed a model that estimated that "the discounted cumulative income earned by the rent controlled building was between 98.5\% and 99.0\% of that earned by the non-rent controlled building".\cite{^haines-2017}

This finding is corroborated by both the Manitoba and the New Jersey study cited above. In New Jersey, median rents in rent controlled cities were found to be roughly the same as rents in non-controlled cities.\cite{^ambrosius-2015-2} In Manitoba, Grant argues "there is no evidence that rent regulations have restricted rents below what would prevail in a perfectly-competitive market under equilibrium conditions".\cite{^grant-2011-2}

Though they do not quite confirm to my criteria above, two other studies are worth mentioning. Frank Denton et al, in a 1993 report commissioned by the CMHC, developed an econometric model and conducted an extensive empirical investigation of the impact of rent controls on Canadian housing markets. They concluded that "there is no evidence that controls influence the long-run rate of increase of rents", nor did they impact housing starts or maintenance though they may lower vacancy rates.\footnote{\label{^denton-1993-note} It's worth noting that this study suffers from the same problems all econometric studies do: the lack of suitable data, the difficulty of adequately modelling housing markets, etc, and the report itself includes many attached comments to that effect.} \cite{^denton-1993} Celia Lazzarin analyzed rent regulations in British Columbia from 1974 to 1984 for her 1990 master's thesis. She found that basically there were too many confounding variables (demographics, unemployment, interest and inflation rates, migration, etc) to attribute the declines in Vancouver's rental supply solely to rent controls.\footnote{\label{^lazzarin-1990-note} BC's controls at the time were rather haphazardly designed (increases were set to 8\% or 10\%, though inflation occasionally exceeded that) hence why I mention it in passing.}\cite{^lazzarin-1990}

\subsection{Controls probably incentivize tenure conversions, though}

There is one noticeable disadvantage to a tenancy rent control: in tight markets the delta between long term tenant rates and market rates can grow rather large. 

Because rents reset between tenants, landlords therefore may try to select for shorter term tenancies (i.e. by preferring students over families) and building smaller units. Lawrence Smith wrote about this in 2003, as well as Richard Arnott.\cite{^smith-2003} \cite{^arnott-2003}

Keeping some rents below market prices has another detrimental effect: it encourages property owners to economically evict their tenants via renovations or to convert to unregulated forms of tenure. At the beginning of this paper, I reviewed Lawrence Smith's 1988 paper which mentions this for Toronto, and earlier I cited McFarlane (2001). I've chosen to highlight three other papers.

Writing in 2017, Martine August documented the financialization of multi family residential properties in Toronto. Aided by the deregulation of capital markets, the end of Canadian social housing provision and historically low interest rates, private equity funds and real estate investment trusts have been buying up aging rental properties -- with the explicitly stated purpose of turfing their existing low-income populations in order to renovate and gentrify their units.\cite{^august-2017}  

David Sims, writing in 2007, examined what happened after Massachusetts ended rent controls in 1994. He argues that units in previously controlled areas became 6-7 percentage points more likely to be rented out, i.e. that units were kept from the rental market.\footnote{\label{sims-2007-note} "My results suggest rent control had little effect on the construction of new housing but did encourage owners to shift units away from rental status and reduced rents substantially." Sims (2007)}\cite{^sims-2007}

Rebecca Diamond et al, in a paper published in the fall of 2017, leveraged a uniquely rich dataset. In 1994, the city of San Francisco extended its rent regulation to buildings with 4 or fewer rental units built before 1980 (about 30\% of the rental stock). Combining a private data provider with property records, they were able to follow individual San Francisco tenants occupying regulated and unregulated housing units from 1994 to the present day. They found "that rent-controlled buildings were almost 10 percent more likely to convert to a condo or a Tenancy in Common".\cite{^diamond-2017}

There is some reason to doubt these numbers prima facie, since converted housing units remain part of the housing stock -- and a large percentage of condominiums get rented out to tenants without this necessarily being reflected in most housing data sources (though it seems that Diamond's dataset mostly controls for this). It may be more accurate to say that primary rental units are being converted to ownership \emph{and} the secondary rental market. 

In addition, California's Ellis Act creates a relatively permissive environment for conversions. Contrast with Massachusetts: Sims found that "rent decontrol is associated with an 8 percentage point increase in the probability of a unit being a condominium", presumably since conversion restrictions were lifted alongside controls.\cite{^sims-2007} Nevertheless, it seems fair to conclude that keeping units below market rates exacerbates the incentive towards selling them or redeveloping them.

I think it is important to distinguish between conversion to ownership tenure and "renovictions". Reading these papers I can't help but think that the actual problem with condo conversions has more to do with rising land values and the lack of new supply. That regular, continuous growth in market rates makes conversions attractive is not surprising. San Francisco's property values have appreciated by 550\% over the last thirty years,\cite{^mclaughlin-2016} and it rather famously doesn't build much in the way of new housing despite creating lots of well paid jobs.\cite{^torres-2017} As seen in Massachusetts, condo conversions can likely be regulated or disincentivized.

With regard to "renovictions", I think we can ameliorate these conditions by moderately controlling vacancies, paired with a temporary exemption for new construction. If inter-tenancy rent increases are restricted by 10 or even 5\% over inflation we reduce the incentive for high tenancy turnover and smooth rapid price increases across the market,\footnote{ August goes a step further and documents how vacancy decontrol in Ontario was likely designed as a predatory move:
\begin{quote}
  In 1998, Residential Equities REIT (ResREIT) was launched, with plans to acquire 27 Ontario properties “to more fully benefit from the anticipated relaxation of rent controls in Ontario, with legislation expected to be proclaimed in force in April 1998”. Quite notably, ResREIT was launched by Dino Chiesa, Assistant Deputy Minister in the Housing Ministry during the time the 1997 Tenant Protection Act (TPA) was conceived. Immediately after, he left public office and started ResREIT to directly profit from new opportunities for rental increases in legislation he helped to craft \cite{^august-2017}
\end{quote}
August's paper makes a compelling argument for the role of (vacancy controlling) rent controls, paired with condominium conversion restrictions, in acting as a kind of stabilizer that disincentivizes predatory gentrification and preserves existing affordable units. I came across it only after I'd written this paper, hence this footnote and quick mention above.}
 while preserving the normal incentive structures and business models previously described.

\subsection{Affordability is about the land}

The essential observation here is that rent controls cause a decline in rentals not because they are rendered unprofitable or unsustainable but because they are crowded out by ownership housing and other, more profitable uses. Today's land prices set the floor on the rent tomorrow's new supply needs to extract, and so it seems to me that when we talk about letting rent prices float to market rates we imply that landlords deserve to capture the growth in value immediately rather than just through capital gains. 

To the extent that increases in land values in certain cities are a function of inelastic supply, capital markets and low interest rates, that seems like a strange reason to support a transfer from tenants to landlords. In fact, allowing landlords to arbitrarily increase prices and extract value gains immediately likely incentivizes existing investors to \emph{prefer} a regime of supply inelasticity -- since no new investment or activity is necessary from their behalf to reap the benefits.

Ultimately, our housing crisis is a matter of income: tenants with low incomes have an \emph{income} problem, not a housing problem.\cite{^hswg-final} \cite{^brescia-2005-3} Land prices and other costs, driven by restrictive land use policies and the speculative bubble, have grown faster than tenant incomes and pushed financial recovery rents beyond what most of the rental population can afford or finds reasonable. While land prices continue to grow above inflation or wages, maintaining affordable housing stock will continue to be a challenge.

 In the absence of subsidy, the private sector is unlikely to build new rental housing for the low end of the market. Though the profitability of building modest cost ownership housing in large volumes can approach that of smaller quantities of high end ownership housing, the same is not true for affordable rentals. It costs only slightly less to build an affordable rental, compared to building high end, but the resulting income stream is substantially smaller.\cite{^hswg-2001}
 
Relatively affordable privately developed housing, then, occurs because as inflation and mortgage payments decrease carrying costs over time the rent necessary to carry a rental investment decreases. And, through a process called filtering, new high-end housing creates vacancies lower in the chain as people move on up to occupy new supply. In theory, as the existing housing stock ages and deteriorates, people with higher incomes will tend to prefer newer, higher quality housing.\cite{^brescia-2005-2} In practice, as housing preferences change and formerly 'downtrodden' areas become trendy, the filtering chain can get interrupted as higher-income people renovate and move into formerly lower-income areas.

\subsection{In review}

Commentators criticizing rent controls often point to New York City as a negative example but that city's experience is rather idiosyncratic. The design and implementation of rent controls varies so much that we must be careful and specific when making comparing different regions.

Price controls are bad in so far that they render the production of goods and services untenable: absent subsidies, a landlord must be able to recoup her investment from the rent she extracts from her tenants. Conversely, as long as landlords can recover increases in their costs over time, their business models should be unaffected. While first-generation controls were likely to be as harmful as described, there is no theoretical reason why a well-designed rent control should disincentivize new construction. This outcome is confirmed by several theoretical and empirical studies.

However, a well-designed rent control allows landlords to pass through capital expenditures and somewhat incentivizes renovations, and therefore does not directly help with gentrification. A well-designed rent control does not disincentivize new supply but it doesn't ease supply inelasticity either -- and therefore by itself cannot ensure affordability.

% While a well-designed rent control is not a major disincentives for new construction supply, they do not That same rent control will, over the long run, allow rents to reach market rates and therefore does not directly help with affordability.Rent controls are not the best tools for preserving affordability or preventing gentrification.\footnote{ Months after writing this paper, I learned of an interesting exception: that of vacancy control applied  }

What \emph{are} rent controls good for, then? That a regulation is relatively neutral does not justify its implementation.

\section{The role of security of tenure}

The economics literature on rent controls has much to say about efficient allocation, property values, maintenance and the supply and demand for rental housing, but unfortunately economists and other commentators rarely seem to have anything to say about security of tenure.\footnote{\label{^jenkins-2009-note} For example, in 2009 Blair Jenkins looked at over fifty economics papers and did not see fit to include anything on security of tenure.\cite{^jenkins-2009}}

The omission is glaring. In a 2003 paper reviewing tenancy rent controls, Richard Arnott noted that:

>Almost all economists lead financially secure lives and were raised by parents who emphasized responsibility and self-discipline. They have little or no personal experience with the insecurity that is ever-present in the lives of the less advantaged—those from dysfunctional families, those not raised to middle-class values, and the less able—who tend to live from one paycheck to the next. Not surprisingly, therefore, most economists ignore or underemphasise the importance of security of tenure in rental housing, even though it is consistently second only to affordability on the list of concerns raised by tenant groups.\cite{^arnott-2003-2}

Security of tenure is the idea that you have the right to occupy your home and be protected from being forced to leave against your will. By way of contrast, a homeowner's right to security of tenure is usually taken for granted. So long as they're current on mortgage payments (if any), taxes, etc, a homeowner is protected from involuntary eviction. That security is not absolute, of course: they may be expropriated or rising interest rates may render them unable to afford their home, but by and large "they cannot be forced out at the whim of someone else".\cite{^yee-1989}

By default in most common law jurisdictions tenants do not have this security. They may be denied a renewal of their lease, they may be subject to seizure by landlords who simply dislike them, or they may be 'economically evicted' due to arbitrary increases in their rent. Providing tenants with security of tenure, i.e. protection from involuntary or arbitrary eviction, requires that we not only ensure that housing units are well-maintained and safe for inhabitation, but also that we prevent landlords from unduly exercising their economic power over tenants. 

Earlier, we examined the theoretical basis for a well-designed rent control, and concluded that it was an ineffective tool for preserving affordability or preventing gentrification. However, rent regulations do seem to be effective at keeping current tenants in their homes. 

For example, consider the case of Massachusetts, which abolished its rent controls in 1994. Four years later, the \emph{Economist} reported that in Cambridge "nearly 40\% of tenants in regulated flats moved out after rent control ended", and that "decontrolled rents overall jumped by more than 50\% between 1994 and 1997".\cite{^economist-1998}  David Sims, writing in 2007 about the same decontrol event, found that "decontrol is associated with a decrease of renter stays of 1.84 years", which is rather "sizeable when compared to the mean renter stay of 6 years in the sample".\cite{^sims-2007} This is framed as a loss of efficiency in terms of labour mobility, but I'm not sure it's that cut and dry. 

Most striking is the result from Rebecca Diamond et al's research. Diamond et al frame rent regulations as a kind of insurance against rent increases whose cost in practice is borne by all tenants, as the restriction in supply causes unregulated or vacant rents to rise more than they would have otherwise. They then found that tenants receiving rent control were up to 20\% likelier to remain in their apartments and that "absent rent control essentially all of those incentivized to stay in their apartments would have otherwise moved out of San Francisco". Diamond et al conclude that the gains in welfare those tenants experience narrowly outweigh the resulting deadweight loss incurred on others, but argue that providing this insurance function directly as a government subsidy or tax credit would be more efficient.\footnote{\label{^practical-insurance}  I'm not sure that insurance is practical. Insurance can either be privately mandated or publically provided. A private mandate is a non-starter; a control at least saves us from the cost of an administrative apparatus. A public provision is more appealing. On the one hand, funding it from the population at large would be more progressive than just across tenants. On the other hand  every public insurance function I can think of eventually acts to cap costs and so it's not clear to me landlords would end up in a significantly different position. That said, a public rent insurance program could be very politically useful: it would re-normalize the routine direct public provision of housing services.}\cite{^diamond-2017}

Given that the welfare gains for San Francisco alone are measured in the billions of dollars, that could be a sizeable intervention. But why shouldn't we intervene? After all, we substantially subsidize private ownership. Its relative attractiveness as an investment is the direct result of government policy. The relative scarcity of land via exclusionary zoning is a government policy. Financial liberalization and the coupling of capital markets to home financing was the result of government policy. 

Most suggestions for how to improve the affordability of rental buildings involve either direct subsidies or the reinstitution of tax shelters, and the extent to which they are built at all today in Canada would not happen without the direct intervention of a government agency, the CMHC. It's not like our housing markets exist in a state of nature.

\subsection{A brief history of housing in Canada}

Property rights and the markets they enable exist to the extent they are enforced and protected by the state. When we establish and regulate rights, we typically seek to balance the interests and concerns of everyone involved, and revisit those tradeoffs as our values and goals shift over time. We think our food should be safe to eat, our doctors should be well trained, and that you shouldn't dump waste anywhere you feel like.

In Canada, it would be difficult to identify a time when we had a completely laissez faire housing market. Some of our earliest municipal bylaws regulated building standards. First, we sought to improve our health, safety, fire and construction standards, and later we gradually began to add a host of land use and development regulations.\cite{^hulchanski-laissez}

These regulations led to the elimination of unhealthy, unsafe and poor quality housing in urban areas. If in 1951 almost one out of every ten houses lacked basic plumbing facilities, by 1982 that had dropped to 1.6\%.\cite{^brescia-2005-4} However, our improved housing standards and growing restrictions on land use led to an increase in the cost of its manufacture. As early as 1914 it became apparent that the private market alone was not providing enough low income housing.\cite{^hulchanski-laissez}

One intervention begat another. Federal incentives were introduced in 1938 to stimulate the development of low income rental housing, and by 1949 the government began to invest directly in its production.\cite{^hulchanski-laissez} Buoyed by the post-war economic and population boom,  we began to seriously expand our welfare state and, concerned with ensuring "enough rental housing production to nourish the golden goose of urban growth", from 1965 to 1995 up to 10\% of all new housing was some mix of social housing.\cite{^suttor-abridged}

These interventions were not limited to the poor; quite the opposite. In 1946, the CMHC was established with the aim of increasing home ownership among the broad middle and lower-middle class. Focusing mainly on making amortized mortgages work for house buyers and private investors in rental housing, by the mid-1960s most households obtained at least part of their mortgage loan directly from the federal government.\cite{^hulchanski-bulletin38}

In fact, most of the history of the role of Canadian government housing policy is an effort to assist ownership. In 2005 alone, more individual homeowners were helped through mortgage insurance than the number of all social housing units funded since the 1970s. And in addition to creating cheaper loans, the federal government also provides subsidies through a variety of tax credits, tax sheltered investment vehicles and tax exemptions. When the federal government began taxing capital gains it exempted the sale of primary residence -- which by 2008 was costing us almost \$6 billion a year in uncollected revenue.\cite{^hulchanski-bulletin38}

In so far that our housing policy has targeted the middle class' standard of living, it has been rather successful. As an investment asset, home ownership confers unique benefits: it provides shelter as well as equity that can be withdrawn later in life. Canadians who pay off their mortgages spend on average only 11\% of their income on housing and, by 1999, the average homeowner earned 208\% more income, and owned 70 times more wealth, than the average tenant.\cite{^hulchanski-bulletin38}

\subsection{Tenants have rights, too}

This is to say: "what kind of living conditions do we want people to enjoy?" and consequently "what, exactly, should be the goal of our housing system?" have been considered important questions for over a century -- and our answers to these questions have shifted over time. We began by regulating the safety of our housing, and today we significantly subsidize its ownership for those who can afford it. 

Similarly, our perception of the nature of the relationship between property owners and tenants has also shifted.\footnote{\label{^yorke-2015-note}  I haven't been able to corroborate this quote but it's eye popping:

\begin{quote}
"The next year, 1969, the Vancouver Tenants Council \textbf{campaigned actively for the right of tenants to vote in civic elections,} for enforcement of the building code, for changes in the Landlord and Tenant Act, for abolition of the Distress Act, and that landlords be compelled to give reasons for evictions."
\end{quote}

I knew that in both Vancouver and Toronto non-resident property owners get to vote (presumably since they pay property taxes) but it blew my mind that this wasn't originally extended to \emph{tenants}!\cite{^yorke-2015}}

Under common law, which concerned itself with a leaseholder's (agricultural) relationship to the land, a landlord was under no statutory requirement to maintain the premises or conduct any repairs -- nor were there any limits on their power to evict or even seize the property of tenants. A review of the applicable laws in 1968 found that landlords possessed such a disparity of bargaining power that tenants did not have a freedom of contract in any real sense.\cite{^hulchanski-tenant-rights}

For a variety of ethical, legal and economic reasons, it became clear that tenants deserved protection, and that applying old-school land law principles to the modern urban apartment rental was totally unsuitable. Gradually the law caught up: Ontario adopted its first residential protection laws in 1970, while the notion that tenants deserve security of tenure was added by 1975. Today, landlords are seen as responsible for providing safe and livable accommodations, and that tenants should be protected from arbitrary evictions.\cite{^yee-1989}

Often, this is framed as a conflict of self-interests between landlords and tenants; tenants suffer disproportionate costs when forced to move, and benefit from stability. In a perfect market, tenants should be on average free from arbitrary increases or poorly maintained units due to the emancipating effect of competition. But in practice, that doesn't seem to describe reality. Given the possibility of economic eviction, the regulation of security of tenure must be accompanied by the regulation of rent.\cite{^hulchanski-1984} David Hulchanski, writing over thirty years ago, compares rent regulations to consumer protection laws: 

\begin{quote}

"Where the rental market cannot function normally, such as in meeting supply, or when moving costs limit the mobility of consumer rental services \[...\] regulations protect consumers who find themselves in inferior bargaining positions".\cite{^hulchanski-1984}
\end{quote}

Every regulation imposes tradeoffs, and in that light we can compare the regulation of rent with the regulation of fire safety. Mandating that landlords' properties satisfy certain minimum fire safety standards also raises costs and therefore diminishes the affordability of housing. Though some are happy to make that macabre argument,\footnote{\label{^mcardle-2017-note} "It’s possible that by allowing large residential buildings to operate without sprinkler systems, the British government has prevented untold thousands of people from being driven into homelessness by higher housing costs. \[...\] Hold these possibilities in mind before condemning those who chose to spend government resources on other priorities. Regulatory decisions are never without costs, and sometimes their benefits are invisible."\cite{^mcardle-2017}} by and large we've decided it's a cost worth bearing: individual people are rarely in position to demand improved construction standards, and fires impose costs on everyone around it. At some point, we will always be dealing with thresholds and equilibriums, and it is up to society to decide what is or isn't acceptable: in Ontario, only about 7 people per million die every year in residential fires.\cite{^fire-deaths}

And, much like fire safety, security of tenure doesn't just benefit individual tenants. The stability provided by security of tenure is the stuff from which well functioning communities are made of.





%------------------------------------------------

%------------------------------------------------

% \section{Conclusion}
% 
% 
% \begin{enumerate}
% \item First numbered list item
% \item Second numbered list item
% \end{enumerate}
% 
% Donec luctus tincidunt mauris, non ultrices ligula aliquam id. Sed varius, magna a faucibus congue, arcu tellus pellentesque nisl, vel laoreet magna eros et magna. Vivamus lobortis elit eu dignissim ultrices. Fusce erat nulla, ornare at dolor quis, rhoncus venenatis velit. Donec sed elit mi. Sed semper tellus a convallis viverra. Maecenas mi lorem, placerat sit amet sem quis, adipiscing tincidunt turpis. Cras a urna et tellus dictum eleifend. Fusce dignissim lectus risus, in bibendum tortor lacinia interdum.

%----------------------------------------------------------------------------------------
%	BIBLIOGRAPHY
%----------------------------------------------------------------------------------------


% \setlength{\parskip}{0em} % reset paragraph spacing
\footnote{\label{myfootnote}Hello footnote}


\begin{thebibliography}{9}

\bibitem{ltb-2017}
 Landlord and Tenant Board. (2017, May). Brochure: A Guide to the Residential Tenancies Act. Retrieved September 29, 2017, from \href{http://www.sjto.gov.on.ca/documents/ltb/Brochures/Guide\%20to\%20RTA\%20(English).html}{sjto.gov.on.ca}

\bibitem{gee-2017}
 Gee, M. (2017, April 20). Rent control isn't the solution to Ontario's housing problem. Retrieved September 12, 2017, from \href{https://beta.theglobeandmail.com/news/toronto/rent-control-isnt-the-solution-to-ontarios-housing-problem/article34753102/}{theglobeandmail.com}

\bibitem{tal-2017}
 Tal, B. (2017, April 4). Rent Control—The Wrong Medicine. Retrieved September 12, 2017, from \href{https://economics.cibccm.com/economicsweb/cds?ID=2595&TYPE=EC_PDF}{economics.cibccm.com}

\bibitem{selley}
 Selley, C. (2017, March 21). Chris Selley: Rent control is a bad solution to the wrong problem. Retrieved September 12, 2017, from \href{http://nationalpost.com/news/toronto/chris-selley-rent-control-is-a-bad-solution-to-the-wrong-problem}{nationalpost.com}

\bibitem{thesun}
 Lafleur, S., \& Filipowicz, J. (2017, April 15). Rent controls wrong answer to housing crisis. Retrieved September 12, 2017, from \href{http://www.torontosun.com/2017/04/15/rent-controls-wrong-answer-to-housing-crisis}{torontosun.com}

\bibitem{cbc-squeeze}
  McGillivray, K. (2017, April 05). Ontario second-worst economy for young people in Canada: report. Retrieved September 12, 2017, from \href{http://www.cbc.ca/news/canada/toronto/generation-squeeze-ontario-economy-1.4054589}{cbc.ca}
  
\bibitem{cbc-martin}
  Martin, S. (2017, February 22). No fixed address: How I became a 32-year-old couch surfer. Retrieved September 29, 2017, from \href{http://www.cbc.ca/news/canada/toronto/no-fixed-address-how-i-became-a-32-year-old-couch-surfer-1.3985771}{cbc.ca}

\bibitem{tgam-jaafari}
  Jaafari, J. D. (2017, April 14). Toronto tenants crushed by rent hike exemptions. Retrieved September 29, 2017, from \href{https://beta.theglobeandmail.com/news/toronto/toronto-tenants-crushed-by-rent-hike-exemptions/article33806825/}{theglobeanmail.com}

\bibitem{mercer-2017}
  Mercer, G. (2017, May 13). Tenants sound alarm on creeping rents. Retrieved October 31, 2017, from \href{https://www.therecord.com/news-story/7312772-tenants-sound-alarm-on-creeping-rents/}{therecord.com}

\bibitem{toronto-2006}
  City of Toronto (2006) \href{https://www1.toronto.ca/city_of_toronto/social_development_finance__administration/files/pdf/housing_rental.pdf}{Rental Housing Supply and Demand Indicators}. Toronto City Planning and Policy Research

\bibitem{smith-1983}
 Smith, L. B. (1983). The Crisis in Rental Housing: A Canadian Perspective. The Annals of the American Academy of Political and Social Science, 465(1), p. 3-4, 58-75. doi:10.1177/0002716283465001006

\bibitem{smith-1988}
  Smith, L. B. (1988). An economic assessment of rent controls: The Ontario experience. The Journal of Real Estate Finance and Economics, 1(3), 217-231. doi:10.1007/bf00658918A



\end{thebibliography}




\bibliographystyle{unsrt}

\bibliography{sample}

\begin{table}[h]
\caption{Ontario housing starts by intended market 1969-2016}
{\footnotesize
\begin{tabular}{@{}>{\bfseries}r r r c r r c c@{}}
\toprule
 & Rentals & Condos & Single + Detached & Other & Total & Non Rentals & Houses + Other \\
\midrule
1969 & 39,897 & 3,586 & 35,484 & 2,479 & 81,446 & 41,549 & 37,963 \\
1970 & 38,561 & 9,881 & 26,201 & 2,032 & 76,675 & 38,114 & 28,233 \\
1971 & 41,945 & 7,652 & 38,483 & 1,900 & 89,980 & 48,035 & 40,383 \\
1972 & 46,134 & 8,427 & 46,169 & 2,203 & 102,933 & 56,799 & 48,372 \\
1973 & 37,047 & 19,794 & 50,701 & 2,994 & 110,536 & 73,489 & 53,695 \\
1974 & 22,260 & 20,920 & 39,944 & 2,379 & 85,503 & 63,243 & 42,323 \\
1975 & 10,394 & 24,309 & 42,212 & 3,053 & 79,968 & 69,574 & 45,265 \\
1976 & 12,457 & 26,992 & 40,754 & 4,479 & 84,682 & 72,225 & 45,233 \\
1977 & 15,402 & 22,020 & 38,263 & 3,445 & 79,130 & 63,728 & 41,708 \\
1978 & 21,105 & 11,781 & 36,556 & 2,268 & 71,710 & 50,605 & 38,824 \\
1979 & 11,938 & 7,328 & 36,160 & 1,461 & 56,887 & 44,949 & 37,621 \\
1980 & 11,642 & 5,164 & 23,321 & 0 & 40,127 & 28,485 & 23,321 \\
1981 & 14,366 & 5,822 & 29,973 & 0 & 50,161 & 35,795 & 29,973 \\
1982 & 15,875 & 2,606 & 19,927 & 100 & 38,508 & 22,633 & 20,027 \\
1983 & 16,647 & 3,325 & 34,967 & 0 & 54,939 & 38,292 & 34,967 \\
1984 & 9,413 & 5,032 & 33,726 & 0 & 48,171 & 38,758 & 33,726 \\
1985 & 13,080 & 6,355 & 45,436 & 0 & 64,871 & 51,791 & 45,436 \\
1986 & 10,774 & 11,950 & 58,746 & 0 & 81,470 & 70,696 & 58,746 \\
1987 & 15,078 & 17,776 & 59,132 & 1,723 & 93,900 & 78,822 & 61,046 \\
1988 & 12,830 & 20,833 & 51,568 & 1,623 & 86,944 & 74,114 & 53,281 \\
1989 & 11,436 & 20,213 & 47,472 & 1,905 & 81,026 & 69,590 & 49,377 \\
1990 & 12,158 & 11,435 & 28,104 & 1,644 & 53,341 & 41,183 & 29,748 \\
1991 & 14,519 & 4,240 & 24,813 & 2,551 & 46,123 & 31,604 & 27,364 \\
1992 & 13,798 & 2,798 & 27,917 & 4,180 & 48,693 & 34,895 & 32,097 \\
1993 & 7,974 & 3,287 & 26,332 & 1,254 & 38,847 & 30,873 & 27,586 \\
1994 & 4,148 & 3,866 & 32,516 & 1,030 & 41,560 & 37,412 & 33,546 \\
1995 & 2,884 & 5,713 & 22,685 & 611 & 31,893 & 29,009 & 23,296 \\
1996 & 1,289 & 6,145 & 31,634 & 444 & 39,512 & 38,223 & 32,078 \\
1997 & 790 & 8,254 & 40,925 & 3 & 49,972 & 49,182 & 40,928 \\
1998 & 1,181 & 9,258 & 39,649 & 0 & 50,088 & 48,907 & 39,649 \\
1999 & 1,323 & 13,316 & 48,246 & 40 & 62,925 & 61,602 & 48,286 \\
2000 & 2,045 & 13,308 & 51,966 & 104 & 67,423 & 65,378 & 52,070 \\
2001 & 2,717 & 16,815 & 50,474 & 256 & 70,262 & 67,545 & 50,730 \\
2002 & 3,886 & 13,244 & 62,305 & 180 & 79,615 & 75,729 & 62,485 \\
2003 & 4,770 & 16,837 & 58,938 & 388 & 80,933 & 76,163 & 59,326 \\
2004 & 3,624 & 18,658 & 57,607 & 5 & 79,894 & 76,270 & 57,612 \\
2005 & 3,843 & 19,836 & 49,400 & 98 & 73,177 & 69,334 & 49,498 \\
2006 & 4,133 & 18,822 & 44,816 & 10 & 67,781 & 63,648 & 44,826 \\
2007 & 2,994 & 14,155 & 45,626 & 0 & 62,775 & 59,781 & 45,626 \\
2008 & 3,867 & 29,443 & 38,613 & 0 & 71,923 & 68,056 & 38,613 \\
2009 & 4,811 & 14,637 & 28,460 & 31 & 47,939 & 43,128 & 28,491 \\
2010 & 3,743 & 17,693 & 35,650 & 18 & 57,104 & 53,361 & 35,668 \\
2011 & 4,785 & 24,959 & 35,466 & 30 & 65,240 & 60,455 & 35,496 \\
2012 & 4,891 & 34,633 & 34,888 & 2 & 74,414 & 69,523 & 34,890 \\
2013 & 3,888 & 23,427 & 31,299 & 0 & 58,614 & 54,726 & 31,299 \\
2014 & 5,053 & 20,045 & 31,064 & 0 & 56,162 & 51,109 & 31,064 \\
2015 & 6,644 & 27,911 & 33,729 & 8 & 68,292 & 61,648 & 33,737 \\
2016 & 7,367 & 25,340 & 39,148 & 8 & 71,863 & 64,496 & 39,156 \\
\bottomrule
\end{tabular}
}
\end{table}



%----------------------------------------------------------------------------------------

\end{document}
