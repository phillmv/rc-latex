%%%%%%%%%%%%%%%%%%%%%%%%%%%%%%%%%%%%%%%%%
% Thin Sectioned Essay
% LaTeX Template
% Version 1.0 (3/8/13)
%
% This template has been downloaded from:
% http://www.LaTeXTemplates.com
%
% Original Author:
% Nicolas Diaz (nsdiaz@uc.cl) with extensive modifications by:
% Vel (vel@latextemplates.com)
%
% License:
% CC BY-NC-SA 3.0 (http://creativecommons.org/licenses/by-nc-sa/3.0/)
%
%%%%%%%%%%%%%%%%%%%%%%%%%%%%%%%%%%%%%%%%%

%----------------------------------------------------------------------------------------
%	PACKAGES AND OTHER DOCUMENT CONFIGURATIONS
%----------------------------------------------------------------------------------------

\documentclass[11pt]{article} % Font size (can be 10pt, 11pt or 12pt) and paper size (remove a4paper for US letter paper)

\usepackage[protrusion=true,expansion=true]{microtype} % Better typography
\usepackage{graphicx} % Required for including pictures
\usepackage{wrapfig} % Allows in-line images
\usepackage[labelfont=bf]{caption}

\usepackage{mathpazo} % Use the Palatino font
\usepackage[T1]{fontenc} % Required for accented characters
\linespread{1.05} % Change line spacing here, Palatino benefits from a slight increase by default

\makeatletter
\renewcommand\@biblabel[1]{\textbf{#1.}} % Change the square brackets for each bibliography item from '[1]' to '1.'
\renewcommand{\@listI}{\itemsep=0pt} % Reduce the space between items in the itemize and enumerate environments and the bibliography

\renewcommand{\maketitle}{ % Customize the title - do not edit title and author name here, see the TITLE block below
\begin{flushright} % Right align
{\LARGE\@title} % Increase the font size of the title

\vspace{50pt} % Some vertical space between the title and author name

{\large\@author} % Author name
\\\@date % Date

\vspace{40pt} % Some vertical space between the author block and abstract
\end{flushright}
}

%----------------------------------------------------------------------------------------
%	TITLE
%----------------------------------------------------------------------------------------

\title{\textbf{Actually, Rent Control Is Great}\\ % Title
Revisiting Ontario's Experience, the Supply of Housing, and Security of Tenure} % Subtitle

\author{\textsc{Phillip Mendonça-Vieira} % Author
\\{\textit{Federation of Metro Tenants Association}}} % Institution

\date{\today} % Date

%----------------------------------------------------------------------------------------

\begin{document}

\maketitle % Print the title section

%----------------------------------------------------------------------------------------
%	ABSTRACT AND KEYWORDS
%----------------------------------------------------------------------------------------

%\renewcommand{\abstractname}{Summary} % Uncomment to change the name of the abstract to something else

\begin{abstract}
Morbi tempor congue porta. Proin semper, leo vitae faucibus dictum, metus mauris lacinia lorem, ac congue leo felis eu turpis. Sed nec nunc pellentesque, gravida eros at, porttitor ipsum. Praesent consequat urna a lacus lobortis ultrices eget ac metus. In tempus hendrerit rhoncus. Mauris dignissim turpis id sollicitudin lacinia. Praesent libero tellus, fringilla nec ullamcorper at, ultrices id nulla. Phasellus placerat a tellus a malesuada.
\end{abstract}

\hspace*{3,6mm}\textit{Keywords:} lorem, ipsum , dolor , sit amet , lectus % Keywords

\vspace{30pt} % Some vertical space between the abstract and first section

%----------------------------------------------------------------------------------------
%	ESSAY BODY
%----- q-----------------------------------------------------------------------------------

\tableofcontents
\setcounter{secnumdepth}{-2}

\let\oldnumberline\numberline% Copy \numberline into \oldnumberline
\renewcommand{\numberline}[1]{\hspace*{-1.5em}}% Remove number argument
\listoffigures
\let\numberline\oldnumberline% Restore \numberline (if needed)
\listoffigures



\section{Introduction}

In April 2017, the government of Ontario decided to extend rent control to every dwelling in the province, as opposed to just buildings constructed before 1991.

Different jurisdictions have overlapping definitions of "rent control". What Ontario engages in is described by some academics and regions as a tenancy rent control, or rent stabilization. It works like this: once a year, your landlord can raise your monthly rent only up to a guideline pegged to inflation, or inflation plus 3\% if their costs have spiked or they made improvements to the unit. Between tenants, landlords are free to price rents at whatever the market will bear.1

This prompted a lot of commentary, ranging from benign skepticism to vigorous condemnation. The chorus sounded like this:

Against all common sense, the province is handing its cities a poisoned chalice: it is textbook economics that price controls sharply reduce the value of new construction.2 Under rent control, the quantity and quality of available rental units will fall as developers are less incentivized to build or invest in rental properties — all of which exacerbates any price crunch.3 The fact is, rent control would largely help high-end renters in a high-end market, since most units built after the rent control exemption are condos. It’s tough to see how rent control would accomplish much except transferring money from unit owners to their tenants.4 Instead, the province should be tackling the root of the problem: the supply of new housing units in Toronto and elsewhere is not keeping up with demand.5

To me this seemed strange, since removing the exemption seemed like a no-brainer. I've been renting in Toronto for almost a decade now, and I've watched rents in my neighbourhood(s) climb substantially. In that period I have moved eight times, and whenever I have looked for housing the uncertainty of living in an uncontrolled apartment weighed heavily on my mind. Much to my disappointment, few if any critics actually addressed what to me felt like the real problem at hand: the lack of security of tenure.

Toronto is in the grips of a housing crisis. A speculative real estate bubble has priced out ownership for low- and middle-income people,6 and it's easy to find stories about sitting tenants seeing their rents jump by hundreds of dollars.7 8 9 When demand for rental units is high and vacancy rates are low, landlords have a lot of power over their tenants — and all the more if they can increase rents at will. The absence of controls allows the unscrupulous to evict tenants exercising their rights, and the eager to extract more for the same service they provided before.

Housing insecurity means you have to move all the time, which is expensive, but the real cost is in the instability it inflicts. The constant possibility of eviction, two months' notice, changes your relationship with the community around you. If your stay is likely to be short, then volunteering at your local school or investing in strong local ties doesn't make a lot of sense. Not only do tenants have a moral right to remain in neighbourhoods they themselves have helped prosper, it is also in our larger economic interest that they do so. If the success of our cities requires us to shift away from homeownership, then we must create the conditions for our communities to succeed with them.

Our housing delivery system is complex. We significantly intervene, subsidize and shape the funding and price mechanisms through which new housing is produced. We also regulate our housing markets in order to correct market inefficiencies, to prevent people from being exploited, and to change incentive structures to deliver outcomes we have deemed necessary or correct. The regulation of rent, like many other kinds of regulation, is a response to market failure.

Rent controls aren't a panacea. They don't create new housing units nor do they divert the speculative investment dollars driving up prices. But critics obscure the fact that rent controls vary markedly from region to region in their design and implementation. Furthermore, a careful look at our housing statistics, and related policy changes, reveals that Ontario's rent controls are likely not at fault for the decline in our supply of purpose-built rental buildings.

If we are indeed serious about tackling our housing crisis and protecting people from suffering from its effects, it's time to face the facts: there are other, far more serious, disincentives and barriers removing supply from our housing markets — and we need to seriously tweak the many ways we intervene in them.

This is the first part in a three part series. In this article, I begin by examining some arguments against rent controls, and revisit the evidence against them in Ontario. After comparing historical data with more recent housing statistics, I then examine two contemporary policy changes that had outsized impacts on our housing markets. The first was the legalization of condominiums, in 1967, and the other was the overhaul of our income tax system in 1972.
%------------------------------------------------

\section{The case against rent control}
\subsection{Vanishing rental supply}

When we talk about rental supply, we typically distinguish between the "primary" rental market, where professional landlords operate purpose built rental buildings, and the "secondary" rental market, where individuals rent out their basement apartments or spare condos.10

We typically favour primary rentals because professional, full-time landlords are more capable of absorbing maintenance costs and are far likelier to provide long-term accommodation. Condo units have a tendency to get flipped, and basement apartments are often vacated for the owner's own use.

The primary rental housing sector has been in a state of crisis for about forty years.11 Beginning in the 1970s, the construction of new private purpose-built rental buildings collapsed. If in 1969 Ontario had 27,543 new, unassisted rental building starts, by the mid 1980s we were building under 5,000 as private developers left the market.12 At first their departure was compensated by government assisted housing starts, but before long the provincial and federal governments began to withdraw funding as well.10

\subsection{Controls in Ontario}

We're blessed that Ontario is a relatively well studied jurisdiction. In a widely cited13 paper written in 1988, Lawrence Smith looked at Ontario's rental market in the aftermath of the province's rent controls.

Around the same time primary rentals dried up, new tenant protection legislation was being introduced and by 1975 rent controls were in effect throughout the province. At the beginning rents were essentially fixed in nominal terms, which in a high inflation era meant their real values quickly declined. New construction was at first exempted, and then not; only by 1986 were the guidelines changed such that rent adjustments became tied to inflation.12

His technical argument against controls goes something like this. Rent control artificially lowers the income that landlords can expect to receive from rental properties. This depresses any motivation investors may have for responding to demand by creating new rental buildings. Meanwhile, lower rent costs relative to ownership encourage more people to stay in the rental market, which creates more demand for fewer units. A control imposed in response to unaffordable rents and low vacancy rates will therefore exacerbate both.

Rent controls don't just affect new construction, and therefore new tenants; they can have stark effects on existing units as well. Faced with a control where increases in rent grow at a rate slower than the costs of maintaining the property, existing landlords are strongly encouraged to let their properties deteriorate, or convert them to condominiums.

Smith argued that all of the above occurred after the province instituted its 1975 rent controls. He found that real rents and capital values of rental units collapsed, and that Toronto lost 11\% of its moderately priced rental housing stock through conversions, demolition and eviction through renovation. By 1986, vacancy rates were an extremely low 0.1\%, but the market was unable to add supply; to quote,

\begin{quote}
In a normally functioning, uncontrolled housing market a vacancy rate below the natural (or equilibrium) rate triggers an increase in real rents and real capital values. This in turn stimulates increased expenditures on the existing stock and increased new construction. Rent controls break this connection between low vacancies and large housing expenditures, and thereby impede the market adjustment necessary to satisfy the excess demand.

…

The timing and severity of the decline in rental housing starts, especially in government unassisted rental starts, and the contrast with the pattern of single detached, semi-detached and duplex starts suggest rent controls substantially reduced the volume of new rental construction in Ontario.

[Other factors such as less favourable demographics, rising interest rates, and increased tenant protection may have exacerbated the decline in rental starts, but rent controls appear to be the primary factor.] 12
\end{quote}

\begin{figure}[ht]
\centering 
\includegraphics[width=\textwidth]{"Ontario Housing Starts 1969-1986".png}
\caption[Ontario housing starts by intended market 1969-1986]{Ontario housing starts by intended market 1969-1986. This graph does not distinguish between private and government assisted rentals: from 1969-1974 private rentals constituted 72\% of all starts, but from 1975 onwards they were under half of all rentals.}
\label{fig:gallery} 
\end{figure}

\section{Revisiting Ontario's experience}
Let's take this for granted, then. Rent controls appear to be the primary factor. Smith's paper was published some time ago. What has happened since?

In 1992, the Rent Control Act limited the kind of capital expenditures landlords could recover via rent increases and once again exempted new rental housing from rent control for a period of 5 years.14 In 1998, the Harris government initiated the most dramatic change since their introduction: capital expenditures could now be fully recovered, vacancy decontrol was introduced for existing units, and rents in new buildings were permanently deregulated.15 16 17

This means that for over twenty years we've lived in a regime where new construction lacked any kind of rent control. If rent controls by themselves are the main disincentive acting on the volume of new rental construction, how do we expect the market to have responded since?

Any resident of Toronto over the past five years can attest to a rapid pace of new construction. My assumption was that, starting from 1998, we should see a slow but steady increase in the rate of construction in new rental buildings to levels similar to those pre-1975.

Compiling data provided by the Canada Mortgage and Housing Corporation (CMHC), I was able to extend figure 1 up to the present day:

\begin{figure}[ht]
\centering 
\includegraphics[width=\textwidth]{"Ontario Housing Starts 1969-2016".png}
\caption[Ontario housing starts by intended market 1969-2016]{Ontario housing starts by intended market 1969-2016. Data from 1987 onwards is restricted to areas with over 10,000 people, and therefore undercounts total starts.}
\label{fig:gallery} 
\end{figure}

Aliquam fringilla non diam sed varius. Suspendisse tellus felis, hendrerit non bibendum ut, adipiscing vitae diam. Lorem ipsum dolor sit amet, consectetur adipiscing elit. Nulla lobortis purus eget nisl scelerisque, commodo rhoncus lacus porta. Vestibulum vitae turpis tincidunt, varius dolor in, dictum lectus. Aenean ac ornare augue, ac facilisis purus. Sed leo lorem, molestie sit amet fermentum id, suscipit ut sem. Vestibulum orci arcu, vehicula sed tortor id, ornare dapibus lorem. Praesent aliquet iaculis lacus nec fermentum. Morbi eleifend blandit dolor, pharetra hendrerit neque ornare vel. Nulla ornare, nisl eget imperdiet ornare, libero enim interdum mi, ut lobortis quam velit bibendum nibh.

Morbi tempor congue porta. Proin semper, leo vitae faucibus dictum, metus mauris lacinia lorem, ac congue leo felis eu turpis. Sed nec nunc pellentesque, gravida eros at, porttitor ipsum. Praesent consequat urna a lacus lobortis ultrices eget ac metus. In tempus hendrerit rhoncus. Mauris dignissim turpis id sollicitudin lacinia. Praesent libero tellus, fringilla nec ullamcorper at, ultrices id nulla. Phasellus placerat a tellus a malesuada.

%------------------------------------------------

\section*{Conclusion}

Fusce in nibh augue. Cum sociis natoque penatibus et magnis dis parturient montes, nascetur ridiculus mus. In dictum accumsan sapien, ut hendrerit nisi. Phasellus ut nulla mauris. Phasellus sagittis nec odio sed posuere. Vestibulum porttitor dolor quis suscipit bibendum. Mauris risus lectus, cursus vitae hendrerit posuere, congue ac est. Suspendisse commodo eu eros non cursus. Mauris ultrices venenatis dolor, sed aliquet odio tempor pellentesque. Duis ultricies, mauris id lobortis vulputate, tellus turpis eleifend elit, in gravida leo tortor ultricies est. Maecenas vitae ipsum at dui sodales condimentum a quis dui. Nam mi sapien, lobortis ac blandit eget, dignissim quis nunc.

\begin{enumerate}
\item First numbered list item
\item Second numbered list item
\end{enumerate}

Donec luctus tincidunt mauris, non ultrices ligula aliquam id. Sed varius, magna a faucibus congue, arcu tellus pellentesque nisl, vel laoreet magna eros et magna. Vivamus lobortis elit eu dignissim ultrices. Fusce erat nulla, ornare at dolor quis, rhoncus venenatis velit. Donec sed elit mi. Sed semper tellus a convallis viverra. Maecenas mi lorem, placerat sit amet sem quis, adipiscing tincidunt turpis. Cras a urna et tellus dictum eleifend. Fusce dignissim lectus risus, in bibendum tortor lacinia interdum.

%----------------------------------------------------------------------------------------
%	BIBLIOGRAPHY
%----------------------------------------------------------------------------------------

\bibliographystyle{unsrt}

\bibliography{sample}

%----------------------------------------------------------------------------------------

\end{document}
