\section{Introduction}

In April, the government of Ontario decided to extend rent control to every dwelling in the province, as opposed to just buildings constructed before 1991.

Different jurisdictions have overlapping definitions of "rent control". What Ontario engages in is described by some academics and regions as a tenancy rent control, or rent stabilization. It works like this: once a year, your landlord can raise your monthly rent only up to a guideline pegged to inflation, or inflation plus 3\% if their costs have spiked or they made improvements to the unit. Between tenants, landlords are free to price rents at whatever the market will bear.\cite{ltb-2017}

This prompted a lot of commentary, ranging from \href{http://tvo.org/article/current-affairs/the-next-ontario/ontario-needs-a-rental-rethink-but-should-tread-carefully}{benign skepticism} to \href{https://beta.theglobeandmail.com/real-estate/toronto/new-ontario-rent-control-rules-exact-opposite-of-what-is-needed-analyst-warns/article34569276/}{vigorous condemnation}. The chorus sounded like this:

Against all common sense, the province is handing its cities a poisoned chalice: it is textbook economics that price controls sharply reduce the value of new construction.\cite{gee-2017}  Under rent control, the quantity and quality of available rental units will fall as developers are less incentivized to build or invest in rental properties --- all of which exacerbates any price crunch.\cite{tal-2017} The fact is, rent control would largely help high-end renters in a high-end market, since most units built after the rent control exemption are condos. It’s tough to see how rent control would accomplish much except transferring money from unit owners to their tenants.\cite{selley} Instead, the province should be tackling the root of the problem: the supply of new housing units in Toronto and elsewhere is not keeping up with demand.\cite{thesun}



\section{The case against rent control}
\subsection{Vanishing rental supply}


\subsection{Controls in Ontario}

\begin{figure}[ht]
\centering 
\includegraphics[width=\textwidth]{"Ontario Housing Starts 1969-1986".png}
\caption[Ontario housing starts by intended market 1969-1986]{Ontario housing starts by intended market 1969-1986. This graph does not distinguish between private and government assisted rentals: from 1969-1974 private rentals constituted 72\% of all starts, but from 1975 onwards they were under half of all rentals.}
\label{fig:gallery} 
\end{figure}

\section{Revisiting Ontario's experience}



\begin{figure}[ht]
\centering 
\includegraphics[width=\textwidth]{"Ontario Housing Starts 1969-2016".png}
\caption[Ontario housing starts by intended market 1969-2016]{Ontario housing starts by intended market 1969-2016. Data from 1987 onwards is restricted to areas with over 10,000 people, and therefore undercounts total starts.}
\label{fig:gallery} 
\end{figure}


