\section{A theoretical overview}
\subsection{The many types of controls}

The point of this paper isn't to say that price controls are a paragon of virtue, but instead to critically examine the received wisdom: the picture is rather murkier than that shown in in the standard textbook analysis.

One major problem with this discussion is that there are actually many qualitatively different kinds of rent controls, which different regions have experimented with over different periods of time. When designing a control, we can decide when and how prices increase, whether they track inflation or another indicator, whether capital expenditures can be recovered, at what rate vacant units can reset to what the market will bear (if at all), and so on.\footnote{\label{^nota-bene-2} For a more exhaustive treatment of the literature, see Lind (2001) and (2003), Brescia (2005), Jenkins (2009), Grant (2011) and Ambrosius et al (2015)}

Richard Arnott, in a widely read paper from 1995, broadly categorizes rent control policy into two distinct phases.\footnote{\label{^arnott-original} This distinction between "first" and "second" generation or "hard" and "soft" controls is not original to Arnott; Hulchanski (1984) also refers to them that way and it's likely that categorization was made contemporaneously when these controls were enacted in the 1970s.} The first wave, or generation, of controls were enacted around World War II and they imposed nominal rent freezes tied to individual units. In a planned war economy that lacked much in the way of private housing construction, this made a certain amount of sense; but afterwards most regions dismantled their controls and only New York City (and some European cities) maintained that wartime policy.\cite{^arnott-1995}

The second generation began in the 1970s as rent control ordinances were passed in Boston, Los Angeles, San Francisco and in a variety of towns in California, Massachussetts, New Jersey, New York, etc. The structure of Canadian governance saw these policies accumulate at the provincial level, and during this period ten provinces also enacted rent controls (though most have since abandoned them). These policies "differed significantly from the first-generation rent control programs" and usually allowed automatic increases in rent, the passing through of additional costs, and other like-minded provisions.\cite{^arnott-1995}

Separating controls into "hard" and "soft" is not especially useful, though. Hans Lind, writing in 2001, goes on to define five distinct functional types of rent control. In his view, regulations can protect sitting tenants from being charged above-market rents (type A), or from increases in rent unattached to increases in costs (type B). Alternatively, regulations can instead bind to units and prevent landlords anywhere from charging above-market rents (type C), prevent rapid inflation by smoothing increases (type D) or, finally, prevent rents from ever reaching actual market prices (type E).\footnote{\label{^note-on-lind}  I am not sure that Lind's categories are especially useful for comparing different regulatory systems, in so far that its abstraction elides too many relevant details. However, it's great for illustrating the wide variety in intent and implementation.} \cite{^lind-2001}

The takeaway here is that when we talk about regulations, we have to be specific since their goals, mechanisms and therefore impacts are going to be different. New York City's rent controls, which Lind identifies as an extreme version of a type E control,\cite{^lind-2001} are the most famous and well studied example, and consequently critics are quick to conflate all rent regulation with the kind experienced there.\footnote{\label{^widely-cited-2} For example, both the \emph{Globe}'s Marcus Gee\cite{^gee-2017} and CIBC's Benjamin Tal\cite{^tal-2017} use the example of New York City when writing their op-eds.} That tendency is unfortunate, since in doing so they perform a sleight of hand: New York City's complex and overlapping rent regulations were enacted at different points of time,\footnote{\label{^nycrgb-2016-note} NYC distinguishes between rent \emph{controls}, which target buildings built before 1947 and continuous tenancy prior to 1971, and rent \emph{stabilization}, which targets buildings built prior to 1974 with rents under \$2,700. Different tenants under different systems have different rights. Rent \emph{controls} may have been a nominal freeze when they were enacted, but today landlords are entitled to a 7.5\% increase per annum.\cite{^nycrgb-2016}\cite{^curbed-2017}} and therefore its experience is idiosyncratic and unlikely to be directly applicable to other cities.

Of course, they're not wrong to criticize nominal rent freezes: obviously, a control regime that over time lowers real income below that of real expenditures is a \emph{bad idea}. It transforms rental properties into endless money pits. It's fine, and likely necessary, to subsidize some or other aspect of how we produce or provide housing units in order to achieve our policy goals -- but it's unreasonable to expect that subsidy to be provided to the exclusive detriment of individual landlords. If investors wish to transfer their wealth to tenants they don't need to go through the trouble of erecting a building.  

But that's a false choice. We're not limited to between an unfettered market and a ruinously restrictive price control.

\subsection{In a competitive market, prices don't increase arbitrarily}

When economists discuss losses in efficiency, allocation and welfare, they're comparing our real world with an idealized "perfectly competitive" market where landlords compete to produce homogeneous housing units, there are no externalities, every actor possesses perfect information, and so on. In this view, a landlord faced with increased demand is free to increase prices and profits accordingly. Abnormally high profits, though, attract other potential landlords who by virtue of adding to the supply of apartments will then drive down their prices.

Given a perfect market any sketch on a napkin will show that if prices are not allowed to rise with demand then new entrants will stay put, supply will not increase, and shortages will follow. However, given perfect competition, the market prices any landlord can fetch will over the long run equal their marginal cost, i.e. the amortized cost of building and operating a housing unit plus the landlord's opportunity cost. Put another way, in a \emph{well functioning market} it's more or less unreasonable to expect the rents any given landlord is able to extract will grow much faster than costs and inflation.\footnote{\label{^arnott-monopoly}  Arnott makes the case that housing markets are actually monopolistically competitive since housing structures and preferences are not homogenous, there are substantial asymmetries in information, and transactions costs are non-trivial. Consequently, rents are set higher than their efficient level and the corresponding deadweight loss can be mitigated (Arnott 2003, p. 106). But for our purposes we don't need to engage with this argument.}

The implication here is that while absolute price ceilings can be and are harmful there is no obvious reason why price \emph{smoothing} such that increases match but do not exceed costs should have a strong effect on the incentive to create new rental housing.\footnote{\label{^arnott-2003-note} "The introduction of tenancy rent control has no obviously strong effect on the incentives to undertake rental housing construction."\cite{^arnott-2003}} Other economists have reached this conclusion. Writing in 2001, Alastair McFarlane developed an econometric model of rent stabilization and concluded that "because allowing fully flexible base rents permits landlords to capture all of the advantages of a rent growth control, neither the timing nor the density of development will be affected by rent stabilization", though landlords are incentivized to redevelop sooner than later.\cite{^macfarlane-2001}

Trivially, investment in multi-residential buildings is a function of one's cost of equity, cost of financing and net operating income; developers may invest in a project expecting significant growth in their net operating income, but in a competitive market that is a rather risky assumption. Therefore, even in the absence of rent controls, projects by and large must be cashflow positive given rents available immediately post-construction.\cite{^black-2012-4}

As far as new supply is concerned we can therefore conceive of a non-harmful rent control: if prices increase with costs and inflation, landlord cashflows should largely be unaffected. It's easy to see why: a cost-adjusted tenancy rent control primarily impacts only one area of the development process: the initial lease up of an empty building.\cite{^tait-2017} Given the long-term nature of multi-residential investment, and provided with the ability to adjust for initial mistakes, over the long run the impact on their finances should be reasonable if not minimal -- and their business model can be satisfied so long as cashflows keep up with costs and capital expenditures.

You don't have to take my word for it. Take GWL Realty Advisors, whose president Paul Finkbeiner was quoted in the \emph{Financial Post}:

\begin{quote}
 “We believe there is a strong demand for rental apartments and this property will lease up over time,” Finkbeiner said about his Livmore project, ... “Apartments provide good long-term returns and very low vacancy levels, it’s just one of the best assets classes from a stability point of view.”
\[...\]
GWL seems to think it can work within the new provincial rules. “As a developer, we are building something that will last for 25-50 years that works for tenants,” said Finkbeiner. “We want long-term renters which is also consistent with our investors that are long-term in nature. These buildings go to pay pensions and people’s investments.”

 He noted Ontario still allows rents to be raised to market level once a tenant leaves a unit and capital improvements to buildings can also be passed on to tenants.

“All we want is a fair rent for our apartments, we do not want above guidelines. We have been able to work within rent controls and still deliver a good product for our investors and tenants,”\cite{^marr-2017}
\end{quote}

\subsection{Evidence on supply from other jurisdictions}

Recall that Ontario's current rent control regime pegs rents to inflation, does not control rents between tenants, and allows cost pass through. Lind categorized it as a type B control,\cite{^lind-2001} and elsewhere it is variously called a kind of tenancy rent control or rent stabilization.

We saw earlier that though Ontario's previous rent control regimes may have been harmful, they were unlikely to be the main disincentives acting on supply -- and especially so since 1998, when vacancies were decontrolled and new construction was exempted entirely. Since we need to compare apples-to-apples, what other evidence can we draw for the impacts of type B rent controls?

Consider Manitoba, whose regulation scheme is broadly similar\footnote{\label{^note-on-mb}  Manitoba does not allow the rent in buildings with more than 3 units to be reset on vacancy, but it exempts new buildings for 20 years and has a more generous cost pass through provision. Grant (2011)} and has regulated rents since roughly 1976. Hugh Grant, writing in 2011, argues that there is no evidence that Manitoba's rent regulation program had a negative effect on the supply of new, or maintenance of existing, rental properties. Manitoba at the time was experiencing a low vacancy rate, which Grant attributed to a rapid influx of immigration, and a relatively inelastic supply due to large planning-to-completion time lags and uncertainty about future rates of population growth.\cite{^grant-2011}

In New Jersey, over one hundred municipalities have enacted their own rent controls. Each city implemented their regulation differently, but by and large they all permit automatic increases, passing on capital improvements, etc; almost half also engage in vacancy decontrol. In 2015 Joshua Ambrosius et al used the 2010 United States Census and compared the regulated cities with unregulated cities. They found that, once they controlled for other factors, New Jersey rent control policies had no statistical impact on rental quality, rental supply, property appreciation or foreclosure rates in the cities that enacted them.\cite{^ambrosius-2015}

In fact, tenancy rent controls seem to barely control rents at all. Earlier this year Graham Haines analyzed Ontario's rent regulations and developed a model that estimated that "the discounted cumulative income earned by the rent controlled building was between 98.5\% and 99.0\% of that earned by the non-rent controlled building".\cite{^haines-2017}

This finding is corroborated by both the Manitoba and the New Jersey study cited above. In New Jersey, median rents in rent controlled cities were found to be roughly the same as rents in non-controlled cities.\cite{^ambrosius-2015-2} In Manitoba, Grant argues "there is no evidence that rent regulations have restricted rents below what would prevail in a perfectly-competitive market under equilibrium conditions".\cite{^grant-2011-2}

Though they do not quite confirm to my criteria above, two other studies are worth mentioning. Frank Denton et al, in a 1993 report commissioned by the CMHC, developed an econometric model and conducted an extensive empirical investigation of the impact of rent controls on Canadian housing markets. They concluded that "there is no evidence that controls influence the long-run rate of increase of rents", nor did they impact housing starts or maintenance though they may lower vacancy rates.\footnote{\label{^denton-1993-note} It's worth noting that this study suffers from the same problems all econometric studies do: the lack of suitable data, the difficulty of adequately modelling housing markets, etc, and the report itself includes many attached comments to that effect.} \cite{^denton-1993} Celia Lazzarin analyzed rent regulations in British Columbia from 1974 to 1984 for her 1990 master's thesis. She found that basically there were too many confounding variables (demographics, unemployment, interest and inflation rates, migration, etc) to attribute the declines in Vancouver's rental supply solely to rent controls.\footnote{\label{^lazzarin-1990-note} BC's controls at the time were rather haphazardly designed (increases were set to 8\% or 10\%, though inflation occasionally exceeded that) hence why I mention it in passing.}\cite{^lazzarin-1990}

\subsection{Some downsides are real, though}

There is one noticeable disadvantage to a tenancy rent control: in tight markets the delta between long term tenant rates and market rates can grow rather large. 

Because rents reset between tenants, landlords therefore may try to select for shorter term tenancies (i.e. by preferring students over families) and building smaller units. Lawrence Smith wrote about this in 2003, as well as Richard Arnott.\cite{^smith-2003} \cite{^arnott-2003} I think we can ameliorate that by moderately controlling vacancies, paired with a temporary exemption for new construction. If inter-tenancy rent increases are restricted by 10 or even 5\% over inflation we reduce the incentive for high tenancy turnover, smooth rapid price increases across the market, and preserve the normal incentive structure described above. However, rent prices over time would still approach market rates.

Keeping some rents below market prices has another detrimental effect: it encourages property owners to economically evict their tenants via renovations or to convert to unregulated forms of tenure. At the beginning of this article, I reviewed Lawrence Smith's 1988 paper which mentions this for Toronto, and earlier I cited McFarlane (2001). I've chosen to highlight two other papers. 

David Sims, writing in 2007, examined what happened after Massachusetts ended rent controls in 1994. He argues that units in previously controlled areas became 6-7 percentage points more likely to be rented out, i.e. that units were kept from the rental market.\footnote{\label{sims-2007-note} "My results suggest rent control had little effect on the construction of new housing but did encourage owners to shift units away from rental status and reduced rents substantially." Sims (2007)}\cite{^sims-2007}

Rebecca Diamond et al, in a paper published in the fall of 2017, leveraged a uniquely rich dataset. In 1994, the city of San Francisco extended its rent regulation to buildings with 4 or fewer rental units built before 1980 (about 30\% of the rental stock). Combining a private data provider with property records, they were able to follow individual San Francisco tenants occupying regulated and unregulated housing units from 1994 to the present day. They found "that rent-controlled buildings were almost 10 percent more likely to convert to a condo or a Tenancy in Common".\cite{^diamond-2017}

There is some reason to doubt these numbers prima facie, since converted housing units remain part of the housing stock -- and a large percentage of condominiums get rented out to tenants without this necessarily being reflected in most housing data sources (though it seems that Diamond's dataset mostly controls for this). It may be more accurate to say that primary rental units are being converted to ownership \emph{and} the secondary rental market. 

In addition, California's Ellis Act creates a relatively permissive environment for conversions. Contrast with Massachusetts: Sims found that "rent decontrol is associated with an 8 percentage point increase in the probability of a unit being a condominium", presumably since conversion restrictions were lifted.\cite{^sims-2007} Nevertheless, it seems fair to conclude that keeping units below market rates exacerbates the incentive towards selling them or redeveloping them.

Reading these papers I can't help but think that the actual problem has more to do with rising land values and the lack of new supply. That regular, continuous growth in market rates makes conversions attractive is not surprising. San Francisco's property values have appreciated by 550\% over the last thirty years,\cite{^mclaughlin-2016} and it rather famously doesn't build much in the way of new housing despite creating lots of well paid jobs.\cite{^torres-2017}

\subsection{Affordability is about the land}

The essential observation here is that rent controls cause a decline in rentals not because they are rendered unprofitable but because they are crowded out by ownership housing and other, more profitable uses. Today's land prices set the floor on the rent tomorrow's new supply needs to extract, and so it seems to me that when we talk about letting rent prices float to market rates we imply that landlords deserve to capture the growth in value immediately rather than just through capital gains. To the extent that increases in land values in certain cities are a function of inelastic supply, capital markets and low interest rates, that seems like a strange reason to support a transfer from tenants to landlords. 

Ultimately, our housing crisis is a matter of income: tenants with low incomes have an \emph{income} problem, not a housing problem.\cite{^hswg-final} \cite{^brescia-2005-3} Land prices and other costs, driven by restrictive land use policies and the speculative bubble, have grown faster than tenant incomes and pushed financial recovery rents beyond what most of the rental population can afford or finds reasonable. While land prices continue to grow above inflation or wages, maintaining affordable housing stock will continue to be a challenge.

 In the absence of subsidy, the private sector is unlikely to build new rental housing for the low end of the market. Though the profitability of building modest cost ownership housing in large volumes can approach that of smaller quantities of high end ownership housing, the same is not true for affordable rentals. It costs only slightly less to build an affordable rental, compared to building high end, but the resulting income stream is substantially smaller.\cite{^hswg-2001}
 
Relatively affordable privately developed housing, then, occurs because as inflation and mortgage payments decrease carrying costs over time the rent necessary to carry a rental investment decreases. And, through a process called filtering, new high-end housing creates vacancies lower in the chain as people move on up to occupy new supply. In theory, as the existing housing stock ages and deteriorates, people with higher incomes will tend to prefer newer, higher quality housing.\cite{^brescia-2005-2} In practice, as housing preferences change and formerly 'downtrodden' areas become trendy, the filtering chain can get interrupted as higher-income people renovate and move into formerly lower-income areas.

\subsection{In review}

Commentators criticizing rent controls often point to New York City as a negative example but that city's experience is rather idiosyncratic. The design and implementation of rent controls varies so much that we must be careful and specific when making comparing different regions.

Price controls are bad in so far that they render the production of goods and services untenable: absent subsidies, a landlord must be able to recoup her investment from the rent she extracts from her tenants. Conversely, as long as landlords can recover increases in their costs over time, their business models should be unaffected. While first-generation controls were likely to be as harmful as described, there is no theoretical reason why a well-designed rent control should disincentivize new construction. This outcome is confirmed by several theoretical and empirical studies.

However, a well-designed rent control allows landlords to pass through capital expenditures and somewhat incentivizes renovations, and therefore does not directly help with gentrification. A well-designed rent control does not disincentivize new supply but it doesn't ease supply inelasticity either -- and therefore by itself cannot ensure affordability.

% While a well-designed rent control is not a major disincentives for new construction supply, they do not That same rent control will, over the long run, allow rents to reach market rates and therefore does not directly help with affordability.Rent controls are not the best tools for preserving affordability or preventing gentrification.\footnote{ Months after writing this paper, I learned of an interesting exception: that of vacancy control applied  }

What \emph{are} rent controls good for, then? That a regulation is relatively neutral does not justify its implementation.
