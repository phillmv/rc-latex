\section{The role of security of tenure}
\subsection{Rent control and the prevention of precarity}

The economics literature on rent controls has much to say about efficient allocation, property values, maintenance and the supply and demand for rental housing, but unfortunately economists and other commentators rarely seem to have anything to say about security of tenure.\footnote{\label{^jenkins-2009-note} For example, in 2009 Blair Jenkins looked at over fifty economics papers and did not see fit to include anything on security of tenure.\cite{^jenkins-2009}}

The omission is glaring. In a 2003 paper reviewing tenancy rent controls, Richard Arnott noted that:

\begin{quote}
Almost all economists lead financially secure lives and were raised by parents who emphasized responsibility and self-discipline. They have little or no personal experience with the insecurity that is ever-present in the lives of the less advantaged—those from dysfunctional families, those not raised to middle-class values, and the less able—who tend to live from one paycheck to the next. Not surprisingly, therefore, most economists ignore or underemphasise the importance of security of tenure in rental housing, even though it is consistently second only to affordability on the list of concerns raised by tenant groups.\cite{^arnott-2003-2}
\end{quote}

Security of tenure is the idea that you have the right to occupy your home and be protected from being forced to leave against your will. By way of contrast, a homeowner's right to security of tenure is usually taken for granted. So long as they're current on mortgage payments (if any), taxes, etc, a homeowner is protected from involuntary eviction. That security is not absolute, of course: they may be expropriated or rising interest rates may render them unable to afford their home, but by and large "they cannot be forced out at the whim of someone else".\cite{^yee-1989}

By default in most common law jurisdictions tenants do not have this security. They may be denied a renewal of their lease, they may be subject to seizure by landlords who simply dislike them, or they may be 'economically evicted' due to arbitrary increases in their rent. Providing tenants with security of tenure, i.e. protection from involuntary or arbitrary eviction, requires that we not only ensure that housing units are well-maintained and safe for inhabitation, but also that we prevent landlords from unduly exercising their economic power over tenants. 

Earlier, we examined the theoretical basis for a well-designed rent control, and concluded that it was an ineffective tool for preserving affordability or preventing gentrification. However, rent regulations do seem to be effective at keeping current tenants in their homes. 

For example, consider the case of Massachusetts, which abolished its rent controls in 1994. Four years later, the \emph{Economist} reported that in Cambridge "nearly 40\% of tenants in regulated flats moved out after rent control ended", and that "decontrolled rents overall jumped by more than 50\% between 1994 and 1997".\cite{^economist-1998}  David Sims, writing in 2007 about the same decontrol event, found that "decontrol is associated with a decrease of renter stays of 1.84 years", which is rather "sizeable when compared to the mean renter stay of 6 years in the sample".\cite{^sims-2007} This is framed as a loss of efficiency in terms of labour mobility, but I'm not sure it's that cut and dry. 

Most striking is the result from Rebecca Diamond et al's research. Diamond et al frame rent regulations as a kind of insurance against rent increases whose cost in practice is borne by all tenants, as the restriction in supply causes unregulated or vacant rents to rise more than they would have otherwise. They then found that tenants receiving rent control were up to 20\% likelier to remain in their apartments and that "absent rent control essentially all of those incentivized to stay in their apartments would have otherwise moved out of San Francisco". Diamond et al conclude that the gains in welfare those tenants experience narrowly outweigh the resulting deadweight loss incurred on others, but argue that providing this insurance function directly as a government subsidy or tax credit would be more efficient.\footnote{\label{^practical-insurance}  I'm not sure that insurance is practical. Insurance can either be privately mandated or publically provided. A private mandate is a non-starter; a control at least saves us from the cost of an administrative apparatus. A public provision is more appealing. On the one hand, funding it from the population at large would be more progressive than just across tenants. On the other hand  every public insurance function I can think of eventually acts to cap costs and so it's not clear to me landlords would end up in a significantly different position. That said, a public rent insurance program could be very politically useful: it would re-normalize the routine direct public provision of housing services.}\cite{^diamond-2017}

Given that the welfare gains for San Francisco alone are measured in the billions of dollars, that could be a sizeable intervention. But why shouldn't we intervene? After all, we substantially subsidize private ownership. Its relative attractiveness as an investment is the direct result of government policy. The relative scarcity of land via exclusionary zoning is a government policy. Financial liberalization and the coupling of capital markets to home financing was the result of government policy. 

Most suggestions for how to improve the affordability of rental buildings involve either direct subsidies or the reinstitution of tax shelters, and the extent to which they are built at all today in Canada would not happen without the direct intervention of a government agency, the CMHC. It's not like our housing markets exist in a state of nature.

\subsection{A brief history of housing in Canada}

Property rights and the markets they enable exist to the extent they are enforced and protected by the state. When we establish and regulate rights, we typically seek to balance the interests and concerns of everyone involved, and revisit those tradeoffs as our values and goals shift over time. We think our food should be safe to eat, our doctors should be well trained, and that you shouldn't dump waste anywhere you feel like.

In Canada, it would be difficult to identify a time when we had a completely laissez faire housing market. Some of our earliest municipal bylaws regulated building standards. First, we sought to improve our health, safety, fire and construction standards, and later we gradually began to add a host of land use and development regulations.\cite{^hulchanski-laissez}

These regulations led to the elimination of unhealthy, unsafe and poor quality housing in urban areas. If in 1951 almost one out of every ten houses lacked basic plumbing facilities, by 1982 that had dropped to 1.6\%.\cite{^brescia-2005-4} However, our improved housing standards and growing restrictions on land use led to an increase in the cost of its manufacture. As early as 1914 it became apparent that the private market alone was not providing enough low income housing.\cite{^hulchanski-laissez}

One intervention begat another. Federal incentives were introduced in 1938 to stimulate the development of low income rental housing, and by 1949 the government began to invest directly in its production.\cite{^hulchanski-laissez} Buoyed by the post-war economic and population boom,  we began to seriously expand our welfare state and, concerned with ensuring "enough rental housing production to nourish the golden goose of urban growth", from 1965 to 1995 up to 10\% of all new housing was some mix of social housing.\cite{^suttor-abridged}

These interventions were not limited to the poor; quite the opposite. In 1946, the CMHC was established with the aim of increasing home ownership among the broad middle and lower-middle class. Focusing mainly on making amortized mortgages work for house buyers and private investors in rental housing, by the mid-1960s most households obtained at least part of their mortgage loan directly from the federal government.\cite{^hulchanski-bulletin38}

In fact, most of the history of the role of Canadian government housing policy is an effort to assist ownership. In 2005 alone, more individual homeowners were helped through mortgage insurance than the number of all social housing units funded since the 1970s. And in addition to creating cheaper loans, the federal government also provides subsidies through a variety of tax credits, tax sheltered investment vehicles and tax exemptions. When the federal government began taxing capital gains it exempted the sale of primary residence -- which by 2008 was costing us almost \$6 billion a year in uncollected revenue.\cite{^hulchanski-bulletin38}

In so far that our housing policy has targeted the middle class' standard of living, it has been rather successful. As an investment asset, home ownership confers unique benefits: it provides shelter as well as equity that can be withdrawn later in life. Canadians who pay off their mortgages spend on average only 11\% of their income on housing and, by 1999, the average homeowner earned 208\% more income, and owned 70 times more wealth, than the average tenant.\cite{^hulchanski-bulletin38}

\subsection{Tenants have rights, too}

This is to say: "what kind of living conditions do we want people to enjoy?" and consequently "what, exactly, should be the goal of our housing system?" have been considered important questions for over a century -- and our answers to these questions have shifted over time. We began by regulating the safety of our housing, and today we significantly subsidize its ownership for those who can afford it. 

Similarly, our perception of the nature of the relationship between property owners and tenants has also shifted.\footnote{\label{^yorke-2015-note}  I haven't been able to corroborate this quote but it's eye popping:

\begin{quote}
"The next year, 1969, the Vancouver Tenants Council \textbf{campaigned actively for the right of tenants to vote in civic elections,} for enforcement of the building code, for changes in the Landlord and Tenant Act, for abolition of the Distress Act, and that landlords be compelled to give reasons for evictions."
\end{quote}

I knew that in both Vancouver and Toronto non-resident property owners get to vote in municipal elections (presumably, since they pay property taxes) but it blew my mind that this wasn't originally extended to \emph{tenants}!\cite{^yorke-2015}}

Under common law, which concerned itself with a leaseholder's (agricultural) relationship to the land, a landlord was under no statutory requirement to maintain the premises or conduct any repairs -- nor were there any limits on their power to evict or even seize the property of tenants. A review of the applicable laws in 1968 found that landlords possessed such a disparity of bargaining power that tenants did not have a freedom of contract in any real sense.\cite{^hulchanski-tenant-rights}

For a variety of ethical, legal and economic reasons, it became clear that tenants deserved protection, and that applying old-school land law principles to the modern urban apartment rental was totally unsuitable. Gradually the law caught up: Ontario adopted its first residential protection laws in 1970, while the notion that tenants deserve security of tenure was added by 1975. Today, landlords are seen as responsible for providing safe and livable accommodations, and that tenants should be protected from arbitrary evictions.\cite{^yee-1989}

Often, this is framed as a conflict of self-interests between landlords and tenants; tenants suffer disproportionate costs when forced to move, and benefit from stability. In a perfect market, tenants should be on average free from arbitrary increases or poorly maintained units due to the emancipating effect of competition. But in practice, that doesn't seem to describe reality. Given the possibility of economic eviction, the regulation of security of tenure must be accompanied by the regulation of rent.\cite{^hulchanski-1984} David Hulchanski, writing over thirty years ago, compares rent regulations to consumer protection laws: 

\begin{quote}
  Where the rental market cannot function normally, such as in meeting supply, or when moving costs limit the mobility of consumer rental services [...] regulations protect consumers who find themselves in inferior bargaining positions.\cite{^hulchanski-1984}
\end{quote}

Every regulation imposes tradeoffs, and in that light we can compare the regulation of rent with the regulation of fire safety. Mandating that landlords' properties satisfy certain minimum fire safety standards also raises costs and therefore diminishes the affordability of housing. Though some are happy to make that macabre argument,\footnote{\label{^mcardle-2017-note} The columnist Meghan McArdle, writing about the Grenfell tower tragedy, callously noted: \begin{quote}It’s possible that by allowing large residential buildings to operate without sprinkler systems, the British government has prevented untold thousands of people from being driven into homelessness by higher housing costs. [...] Hold these possibilities in mind before condemning those who chose to spend government resources on other priorities. Regulatory decisions are never without costs, and sometimes their benefits are invisible.\cite{^mcardle-2017}\end{quote}} by and large we've decided it's a cost worth bearing: individual people are rarely in position to demand improved construction standards, and fires impose costs on everyone around it. At some point, we will always be dealing with thresholds and equilibriums, and it is up to society to decide what is or isn't acceptable: in Ontario, only about 7 people per million die every year in residential fires.\cite{^fire-deaths}

And, much like fire safety, security of tenure doesn't just benefit individual tenants. The stability provided by security of tenure is the stuff from which well functioning communities are made of.

\subsection{Security of tenure is a public good}

In economics there exists a concept of a "public good", which broadly applies to any "service" or "thing that people derive benefit from" that is both non-excludable and non-rivalrous. Non-excludable means that you can't prevent people from enjoying it; non-rivalrous means that by enjoying the good you are not depriving other people from doing the same.\cite{^cowen-2007}

The classical examples include clean air, national security, knowledge, and so on. I can't prevent you from breathing in the same fresh air that I'm enjoying, and when you consume official government statistics you are not depriving me of the same benefit. Public goods aren't necessarily provided by the state, much the same way that the state provides access to some private goods. The distinction is relevant because the nature of public goods incentivizes people to be jerks and therefore public goods "must be provided for everyone if they are to be provided for anyone, and they must be paid for collectively or they cannot be had at all".\cite{^galea-2016}

Though housing itself is a private good, one's enjoyment of security does not impinge on the security of others. As we've seen, security of tenure cannot be provided without a collective investment of some sort since individual tenants rarely possess the bargaining power to afford it themselves. But more importantly, security of tenure provides benefits to the community at large -- and in that vein also suffers from the free-rider problem that plagues other public goods.

\subsection{Security of tenure is good for the public}

Our communities consist of both physical infrastructure, like roads and public transit, as well as "human" infrastructure, like community organizations and loose neighbourhood ties. A neighbourhood may be desirable for its proximity to a grocery store -- but also for the quality of its schools, the absence of crime and the vitality of its community events. A healthy neighbourhood is one where its residents are looking out for one another.

When they raise rents above costs, landlords are in effect cashing in on new or improved local amenities provided either by the state or the local community itself. The reduction of crime, or the expansion of public transit, readily provide rationales for increasing rents. Creating conditions where tenants may reasonably fear investment in their communities is a rather perverse outcome. It doesn't stop there.

Tenants face substantial hurdles. I recently read Matthew Desmond's incredible and harrowing book "Evicted", in which he followed several low-income renters in the city of Milwaukee over the course of a year. Early on, he writes:

\begin{quote}

"The public peace--the sidewalk and street peace--of cities is not kept primarily by the police, necessary as police are. It is kept primarily by an intricate, almost unconscious, network of voluntary controls and standards among the people themselves, and enforced by the people themselves." So wrote Jane Jacobs in \emph{The Death and Life of Great American Cities}. Jacobs believed that a prerequisite for this type of healthy and engaged community was the presence of people who were simply present, who looked after the neighborhood.  She has been proved right: disadvantaged neighborhoods with higher levels of "collective efficacy"--the stuff of loosely linked neighbors who trust one another and share expectations about how to make their community better--have lower crime rates.
\end{quote}

Towards the end of the book, he adds:

\begin{quote}

Residential stability begets a kind of psychological stability, which allows people to invest in their home and social relationships. It begets school stability, which increases the chances that children will excel and graduate. And it begets community stability, which encourages neighbours to form strong bonds and take care of their block. But poor families enjoy little of that because they are evicted at such high rates. 
\[...\]
Eviction even affects the communities that displaced families leave behind. Neighbors who cooperate with and trust one another can make their streets safer and more prosperous. But that takes time. Efforts to establish local cohesion and community investment are thwarted in neighborhoods with high turnover rates. In this way, eviction can unravel the fabric of a community, helping to ensure that neighbors remain strangers and that their collective capacity to combat crime and promote civic engagement remains untapped.\cite{^desmond}
\end{quote}

Faced with the unpredictable but certain need to move in the near future, it's harder to establish deep ties to an neighbourhood. As a result, tenants are routinely discounted or ignored by our political structures,\footnote{\label{^mcgrath-2017-note} As the columnist John McGrath wryly noted, \begin{quote}If you don’t currently own a house in Toronto, preferably a detached one, the city’s political class doesn’t care about you and doesn’t even really want you.\cite{^mcgrath-2017}\end{quote}} and face greater hardships accessing necessary social services and economic opportunities.

In 2016, the New Zealand Housing Foundation commissioned Charles Waldegrave et al to undergo a literature review of the "Social and Economic Impacts of Housing Tenure" and the results are rather bleak. Being a homeowner, as opposed to a tenant, means you live longer and are both physically and psychologically healthier. It makes you less likely to retire early due to health reasons and homeowners on average spend less time unemployed. Owners live in neighbourhoods with lower crime rates, and their children are less likely to suffer from depression, and are more likely to graduate from high school.\cite{^waldegrave-2016}

Some of these effects are undoubtedly due to homeowners' higher levels of income and wealth. Life is much less stressful if you have the financial cushion to weather various misfortunes. However, these effects mostly persist even after controlling for socio-economic status and/or income. Waldegrave et al note that their study is limited, and did not include studies that focused on mortgage and rent stress though "it is acknowledged that unaffordable housing of whatever tenure type will almost certainly lead to negative health and social outcomes".\cite{^waldegrave-2016} A few of the studies they encountered did not find meaningful differences once they accounted for \emph{residential stability} -- and, after income, that instability is arguably the main source of stress differentiating owners from tenants.

In practical terms, it's hard to avoid the inference that being a tenant is a rather harsh externality that our housing policies impose largely on the poor and the recent immigrant.

\section{Conclusions}

For better or for worse, renting is the future. 

A popular urbanist school of thought posits that our cities are the economic engines of the future. Globalization diminished the importance of physical distance in the manufacturing economy, but the resulting shift to the service economy has magnified the importance of clustering effects and economies of scale provided by greater densities of goods and people -- to say nothing about climate change and the environmental unsustainability of our low density suburbs.

In this view, it's not hard to imagine a near future where the equity requirements of residential ownership (let alone freehold structures) has priced out most people. Back in March, CIBC's Benjamin Tal wrote that "the GTA market is fast approaching a full-blown affordability crisis" as a surge in demand due to "a notable increase in speculative and flipping activity" is pricing out most people from ownership markets. He argues that therefore municipalities must "rethink the role of rental activity in the region’s housing mix".\cite{^tal-rent-2017} Though I am prone to quibble with some of his preferred solutions,\footnote{ Relaxing intensification targets and eating away at the Greenbelt seems short sighted. It says a lot that in terms of our political economy giving up those goals is easier than rezoning our land and increasing density. However, it will harm us in the long run as we miss out on economic clustering effects and continue to waste good money on unsustainable low density development. While I'm at it, the implicit assumption that the present equilibrium, where land prices are allowed to inflate at arbitrary values, is value neutral or not worth direct policy action is suspect but then again, I don't work at a mortgage lender.} it seems correct to observe that current macro economic conditions are rushing us towards a new era of unobtainable ownership prices.

In 2011, a little under half of the population of the city of Toronto rented their housing,\cite{^census-2011} and the preliminary results from the 2016 census suggest renting is poised for a comeback: rental housing dominates recent growth and change, and home-ownership is now out of reach for the young and the middle- and low-income.\cite{^suttor-2017} It's become fun for real estate commentators to shrug and joke about life being unfair,\footnote{ Rob Carrick, reprinting in \emph{The Globe and Mail} the quote: "The housing situation in Toronto is never going to be fair, but then again, life isn't fair either."\cite{^carrick-2017}} but our housing universe and its financing mechanisms aren't just some random happenstance -- they're the result of decisions and policies we have made over the decades.


Critics do us a disservice by pretending rent controls are about affordability or supply. The best rationale for instituting rent controls is that of ensuring security of tenure, and in an environment where rents are rapidly increasing it becomes necessary to regulate rent. Not only do tenants deserve security of tenure, but we lead healthier, more productive lives in safer, more pleasant communities when residential stability can be taken for granted. 

A close examination of recent construction starts, and policy changes affecting the economics of purpose built rental buildings, reveals that Ontario's rent regulations are likely not the main disincentive removing supply from our housing markets. The legalization of condominiums in 1967 and the federal government's overhaul of the Income Tax Act in 1972 (amongst other tax changes) had a big impact on the relative profitability of rental buildings, and are likely the cause of their decline relative to other kinds of housing.

Almost no two rent controls are the same, and you have to be very careful when comparing different cities or regions. While earlier rent controls were poorly designed, we've since learned and rent controls today are usually adequately designed to prevent most negative effects. There is evidence from other jurisdictions with similar kinds of rent controls to that of Ontario's current regime that shows a lack of effect on the supply of new housing. Anyone who when discussing rent controls does not bring up the problem of security of tenure, or blithely compares regulations from different jurisdictions, may be pulling your leg.


Finally, housing is complicated and our governments routinely intervene and shape its outcomes. They did a great job creating and subsidizing suburbs and middle class ownership, but they are currently failing the poor and people who live in cities.

Long term tenants have a legitimate interest in staying in the communities they themselves have made successful. Why should they be subjected to a tenure regime that significantly disadvantages them socially, economically and politically compared to the subsidized homeowning population?

